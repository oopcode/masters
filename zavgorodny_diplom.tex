\documentclass[a4paper, titlepage, 12pt]{article}
\usepackage[utf8]{inputenc}
\usepackage[russian]{babel}
\usepackage{mathtools}
\usepackage{fixltx2e}
\usepackage{stmaryrd}
\usepackage[round, sort]{natbib}
\usepackage[nottoc,numbib]{tocbibind}
\usepackage{gb4e}
\noautomath

% COMPILE WITH: reset && pdflatex zavgorodny_diplom.tex && bibtex zavgorodny_diplom && pdflatex zavgorodny_diplom.tex && pdflatex zavgorodny_diplom.tex

\title{Семантика и дистрибуция фокусных частиц в русском языке}

\author{Андрей Завгородний}

\renewcommand*\contentsname{Summary}

\begin{document}

\begin{titlepage}

\newcommand{\HRule}{\rule{\linewidth}{0.5mm}} % Defines a new command for the horizontal lines, change thickness here

\center % Center everything on the page
 
%----------------------------------------------------------------------------------------
%   HEADING SECTIONS
%----------------------------------------------------------------------------------------

\textsc{\large Московский государственный университет имени \\ М.В. Ломоносова}\\[1.5cm] % Name of your university/college
\textsc{\large Филологический факультет}\\[0.5cm] % Major heading such as course name
\textsc{\large Отделение фундаментальной и прикладной лингвистики}\\[0.5cm] % Minor heading such as course title

~\\[2.0cm]

%----------------------------------------------------------------------------------------
%   TITLE SECTION
%----------------------------------------------------------------------------------------


{ \huge Семантика и дистрибуция фокусных частиц в русском языке}\\[0.4cm] % Title of your document
 
%----------------------------------------------------------------------------------------
%   AUTHOR SECTION
%----------------------------------------------------------------------------------------

~\\[3.0cm]

\begin{minipage}{0.4\textwidth}
\begin{flushleft} \large
Дипломная работа студента II курса магистратуры Андрея Олеговича Завгороднего \\
\end{flushleft}
\end{minipage}
~
\begin{minipage}{0.4\textwidth}
\begin{flushright} \large
Научный руководитель д.ф.н., проф. \\  Сергей Георгиевич Татевосов \\
\end{flushright}
\end{minipage}\\[5cm]

% If you don't want a supervisor, uncomment the two lines below and remove the section above
%\Large \emph{Author:}\\
%John \textsc{Smith}\\[3cm] % Your name

%----------------------------------------------------------------------------------------
%   DATE SECTION
%----------------------------------------------------------------------------------------

{\large Москва, 2017} % Date, change the \today to a set date if you want to be precise

%----------------------------------------------------------------------------------------
%   LOGO SECTION
%----------------------------------------------------------------------------------------

%\includegraphics{Logo}\\[1cm] % Include a department/university logo - this will require the graphicx package
 
%----------------------------------------------------------------------------------------

% \vfill % Fill the rest of the page with whitespace

\end{titlepage}

\thispagestyle{empty} 
\tableofcontents
\thispagestyle{empty}

\clearpage


\section[Введение]{Введение}

Введение.

\setcounter{page}{1}

\section{Фокус: семантика и прагматика}

В данной главе мы опишем наиболее значимые подходы к анализу семантики и прагматики фокуса. 

\subsection{Определение и общие сведения}

Фокус --- грамматическая категория, выделяющая в высказывании информационный компонент, являющийся новым или важным в том смысле, что говорящий не считает его разделенным между собой и слушающим \citep{Jackendoff1972}.

\medskip

Фокус может выражаться при помощи просодических (\ref{pitchAccentF}, фразовое ударение), синтаксических (\ref{cleftF}, клефт\footnote{Мнения относительно клефта как формы фокуса расходятся; см. \citep{ward2002grammar}}) или морфологических средств (\ref{morphemeF}), а также их комбинаций:

\begin{exe}
    \ex
    \begin{xlist}
        \ex \label{pitchAccentF} Я ищу \textbf{Машу}.
        \ex \label{cleftF} It is \textbf{John} we are looking for.
        \ex \label{morphemeF}
            {\footnotesize West Chadic}
            \gll Tí bà wúm-\textit{á} \textbf{kwálíngálá}. \\
                 \textsc{3sg} \textsc{prog} chew-\textsc{foc} colanut \\
            \glt `He is chewing \textbf{colanut}.' \citep[ex.\ 3b]{hartmann}
    \end{xlist}
\end{exe}

\medskip

Фокус имеет прямое отношение к семантике высказывания, поскольку способен влиять на его истинностное значение. Ниже приведен классический пример смыслоразличительной функции фокуса из английского языка:

\begin{exe}
    \ex \begin{xlist} \label{truthValues}
        \ex \label{truthValues1} John only introduced [\textsc{Bill}]\textsubscript{F} to Sue.
        \ex \label{truthValues2} John only introduced Bill to [\textsc{Sue}]\textsubscript{F}.
    \end{xlist}
\end{exe}

В примере (\ref{truthValues1}) утверждается, что единственным человеком, которого представили Сью, был Билл. В (\ref{truthValues2}), напротив, говорится, что Билла представили только Сью (в то время как Сью могли также представить Джона, Гарри, etc.)

\medskip

Фокус также может влиять на уместность высказывания в дискурсе:

\begin{exe}
    \ex \begin{xlist} \label{discourseSensitivity}
        \ex Эту книгу продают на Озоне.
        \ex Ну и что? Эту книгу продают даже в [\textsc{Читай-городе}]\textsubscript{F}.
        \ex [\#]{Ну и что? Эту книгу даже [\textsc{продают}]\textsubscript{F} в Читай-городе.}
    \end{xlist}
\end{exe}

\medskip

Здесь и далее мы будем выделять часть высказывания, несущую на себе фразовое ударение, \textsc{малыми заглавными} буквами; составляющая, которая несет на себе фокус, будет обозначаться нижним индексом F. Указанное разграничение неслучайно: фразовое ударение в одном и том же месте высказывания может интерпретироваться как фокус на составляющих разного размера:

\begin{exe}
    \ex \begin{xlist} \label{pitchAccentVSfocus}
        \ex Марк купил книгу о [\textsc{шахматах}]\textsubscript{F}. \\ \textit{(О чем была книга, которую купил Марк?)}
        \ex Марк купил книгу [о \textsc{шахматах}]\textsubscript{F}. \\ \textit{(Какую книгу купил Марк?)}
        \ex Марк купил [книгу о \textsc{шахматах}]\textsubscript{F}. \\ \textit{(Что купил Марк?)}
        \ex Марк [купил книгу о \textsc{шахматах}]\textsubscript{F}. \\ \textit{(Что сделал Марк?)}
        \ex ~[Марк купил книгу о \textsc{шахматах}]\textsubscript{F}. \\ \textit{(Что произошло?)}
    \end{xlist}
\end{exe}

\medskip

Проблеме различных стратегий проекции фокуса посвящена обширная литература. Существуют две основных теории реализации фокуса на составляющей: проекции \textit{сверху} и проекции \textit{снизу} \citep{winkler1997focus}. В первом случае \citep{gussenhoven1983focus} утверждается, что признак фокуса присваивается наиболее высокому узлу и затем просачивается ``вниз'', пока не находит наиболее подходящее место для реализаци; во втором случае \citep{selkirk1984phonology, selkirk1996prosodic} признак фокуса изначально находится на составляющей с фразовым ударением, после чего проецируется вверх, маркируя сферу действия фокуса.

\subsection{Семантика фокуса}

В данном разделе мы обсудим подходы к анализу семантики фокуса. В \ref{alternativeSemantics} мы изложим основные положения \textit{семантики альтернатив} \citep{rooth1985association,rooth1992theory}; в \ref{structuredMeanings} мы рассмотрим иной подход, названный теорией \textit{упорядоченных значений} \citep{Cresswell1985}, а также проведем его сравнение с семантикой альтернатив.

\subsubsection{Семантика альтернатив \citep{rooth1985association,rooth1992theory}} \label{alternativeSemantics}

Семантика альтернатив \citep{rooth1985association,rooth1992theory} является одним из наиболее известных и распространенных подходов к анализу семантики фокуса. Утверждается, что фокус привносит в процесс интерпретации высказывания \textit{множество альтернатив}, существующих параллельно с ``обычной'' семантикой высказывания.

\medskip

Этот подход во многом похож на анализ семантики вопросов в работах \citep{hamblin1973questions,karttunen1977syntax,groenendijk1985semantics}, где интерпретация выражения \textit{What does Andrew smoke?} будет выглядеть как множество альтернатив $ \{ smoke(Andrew, x) \ | \ object(x) \} $. Данное множество --- то, что в \citep{rooth1985association} называется \textsc{фокусным значением} (англ. \textit{fucus semantic value}).

\begin{exe}
    \ex $ \llbracket \textsc{only} \rrbracket = \lambda P_{<e,t>}.\lambda x_{e}.[ \forall Q_{<e,t>} [Q(x) \land C(Q)] \rightarrow Q = P ]$
\end{exe}

\subsubsection{Структурированные значения \citep{Cresswell1985}} \label{structuredMeanings}

\subsection{Прагматика фокуса}

\subsection{Теория QFC \citep{beaver2008sense}} \label{qfcTheory}

\section{Фокус и фокусные операторы}

Смыслоразличительные (или релевантные для дискурса) свойства фокуса наблюдаются в присутствии особых выражений, которые называются \textit{чувствительными к фокусу}; к таким элементам (среди прочих) относятся:

\begin{itemize}\setlength\itemsep{0em}
    \item[---] Наречия единственности: \textit{только,  лишь, ...;}
    \item[---] Нескалярные аддитивные операторы: \textit{тоже, ...;}
    \item[---] Скалярные аддитивные операторы: \textit{даже, аж, ...;}
    \item[---] Операторы отрицания: \textit{не, нет;} etc.
\end{itemize}

\newpage
\bibliography{refs} 
\bibliographystyle{plainnat}

\end{document}
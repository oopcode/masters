\documentclass[a4paper, titlepage, 14pt]{article}
\usepackage[utf8]{inputenc}
\usepackage[russian]{babel}
\usepackage{mathtools}
\usepackage{extsizes}
\usepackage{fixltx2e}
\usepackage{stmaryrd}
\usepackage[perpage]{footmisc}
\usepackage{bm}
\usepackage{array}
\usepackage[table]{xcolor}
\usepackage{tikz}
\usepackage{float}
\usepackage{qtree}
\usepackage{pst-tree}
\usepackage[round, sort]{natbib}
\usepackage[nottoc,numbib]{tocbibind}
\usepackage{gb4e}

\noautomath
\restylefloat{table}
\renewcommand{\baselinestretch}{1.2} 
\renewcommand{\arraystretch}{1.5}
\newcommand*{\LargerCdot}{\raisebox{-0.25ex}{\scalebox{1.2}{$\cdot$}}}
\newcommand*{\LargeCdot}{\raisebox{-0.25ex}{\scalebox{1.8}{$\cdot$}}}
\newcolumntype{C}[1]{>{\centering\arraybackslash\hspace{0pt}}m{#1}}
\def\mathhyphen{{\hbox{-}}}


% COMPILE WITH: reset && pdflatex zavgorodny_diplom.tex && bibtex zavgorodny_diplom && pdflatex zavgorodny_diplom.tex && pdflatex zavgorodny_diplom.tex

\title{Семантика и дистрибуция фокусных частиц в русском языке}

\author{Андрей Завгородний}

\renewcommand*\contentsname{Summary}

\begin{document}

\begin{titlepage}

\newcommand{\HRule}{\rule{\linewidth}{0.5mm}} % Defines a new command for the horizontal lines, change thickness here

\center % Center everything on the page
 
%----------------------------------------------------------------------------------------
%   HEADING SECTIONS
%----------------------------------------------------------------------------------------

\textsc{\large Московский государственный университет имени \\ М.В. Ломоносова}\\[1.5cm] % Name of your university/college
\textsc{\large Филологический факультет}\\[0.5cm] % Major heading such as course name
\textsc{\large Отделение фундаментальной и прикладной лингвистики}\\[0.5cm] % Minor heading such as course title

~\\[2.0cm]

%----------------------------------------------------------------------------------------
%   TITLE SECTION
%----------------------------------------------------------------------------------------


{ \huge Семантика и дистрибуция скалярных фокусных частиц в русском языке}\\[0.4cm] % Title of your document
 
%----------------------------------------------------------------------------------------
%   AUTHOR SECTION
%----------------------------------------------------------------------------------------

~\\[3.0cm]

\begin{minipage}{0.4\textwidth}
\begin{flushleft} \large
Дипломная работа студента II курса магистратуры Андрея Олеговича Завгороднего \\
\end{flushleft}
\end{minipage}
~
\begin{minipage}{0.4\textwidth}
\begin{flushright} \large
Научный руководитель д.ф.н., проф. \\  Сергей Георгиевич Татевосов \\
\end{flushright}
\end{minipage}\\[5cm]

% If you don't want a supervisor, uncomment the two lines below and remove the section above
%\Large \emph{Author:}\\
%John \textsc{Smith}\\[2.5cm] % Your name

%----------------------------------------------------------------------------------------
%   DATE SECTION
%----------------------------------------------------------------------------------------

{\large Москва, 2017} % Date, change the \today to a set date if you want to be precise

%----------------------------------------------------------------------------------------
%   LOGO SECTION
%----------------------------------------------------------------------------------------

%\includegraphics{Logo}\\[1cm] % Include a department/university logo - this will require the graphicx package
 
%----------------------------------------------------------------------------------------

% \vfill % Fill the rest of the page with whitespace

\end{titlepage}

\thispagestyle{empty} 
\tableofcontents
\thispagestyle{empty}

\clearpage


\section[Введение]{Введение}

Введение.

\setcounter{page}{1}

\section{Фокус: семантика и прагматика}

В данной главе мы опишем наиболее значимые подходы к анализу семантики и прагматики фокуса. 

\subsection{Определение и общие сведения}

Фокус --- грамматическая категория (признак), выделяющая в высказывании информационный компонент, являющийся новым или важным в том смысле, что говорящий не считает его разделенным между собой и слушающим \citep{Jackendoff1972}. Данное определение, разумеется, не может считаться исчерпывающим; ниже мы попытаемся привести необходимую и достаточную информацию о способах выражения и основных случаях употребления фокуса.

\medskip

Фокус может выражаться при помощи просодических (\ref{pitchAccentF}, фразовое ударение), синтаксических (\ref{cleftF}, клефт\footnote{Мнения относительно клефта как формы фокуса расходятся; см. \citep{ward2002grammar}}) или морфологических средств (\ref{morphemeF}), а также их комбинаций:

\begin{exe}
    \ex
    \begin{xlist}
        \ex \label{pitchAccentF} Я ищу \textbf{Машу}.
        \ex \label{cleftF} It is \textbf{John} we are looking for.
        \ex \label{morphemeF}
            {\footnotesize West Chadic}
            \gll Tí bà wúm-\textit{á} \textbf{kwálíngálá}. \\
                 \textsc{3sg} \textsc{prog} chew-\textsc{foc} colanut \\
            \glt `He is chewing \textbf{colanut}.' \citep[ex.\ 3b]{hartmann}
    \end{xlist}
\end{exe}

\medskip

Фокус имеет прямое отношение к семантике высказывания, поскольку способен влиять на его истинностное значение. Ниже приведен классический пример смыслоразличительной функции фокуса из английского языка:

\begin{exe}
    \ex \begin{xlist} \label{truthValues}
        \ex \label{truthValues1} John only introduced [\textsc{Bill}]\textsubscript{F} to Sue.
        \ex \label{truthValues2} John only introduced Bill to [\textsc{Sue}]\textsubscript{F}.
    \end{xlist}
\end{exe}

В примере (\ref{truthValues1}) утверждается, что единственным человеком, которого представили Сью, был Билл. В (\ref{truthValues2}), напротив, говорится, что Билла представили только Сью (в то время как Сью могли также представить Джона, Гарри, etc.)

\medskip

Фокус также может влиять на уместность высказывания в дискурсе:

\begin{exe}
    \ex \begin{xlist} \label{discourseSensitivity}
        \ex Эту книгу продают на Озоне.
        \ex Ну и что? Эту книгу продают даже в [\textsc{Читай-городе}]\textsubscript{F}.
        \ex [\#]{Ну и что? Эту книгу даже [\textsc{продают}]\textsubscript{F} в Читай-городе.}
    \end{xlist}
\end{exe}

\medskip

Здесь и далее мы будем выделять часть высказывания, несущую на себе фразовое ударение, \textsc{малыми заглавными} буквами; составляющая, которая несет на себе фокус, будет обозначаться нижним индексом F. Указанное разграничение неслучайно: фразовое ударение в одном и том же месте высказывания может интерпретироваться как фокус на составляющих разного размера:

\begin{exe}
    \ex \begin{xlist} \label{pitchAccentVSfocus}
        \ex Марк купил книгу о [\textsc{неомарксизме}]\textsubscript{F}. \\ \textit{(О чем была книга, которую купил Марк?)}
        \ex Марк купил книгу [о \textsc{неомарксизме}]\textsubscript{F}. \\ \textit{(Какую книгу купил Марк?)}
        \ex Марк купил [книгу о \textsc{неомарксизме}]\textsubscript{F}. \\ \textit{(Что купил Марк?)}
        \ex Марк [купил книгу о \textsc{неомарксизме}]\textsubscript{F}. \\ \textit{(Что сделал Марк?)}
        \ex ~[Марк купил книгу о \textsc{неомарксизме}]\textsubscript{F}. \\ \textit{(Что произошло?)}
    \end{xlist}
\end{exe}

\medskip

Проблеме различных стратегий проекции фокуса посвящена обширная литература. Существуют две основных теории реализации фокуса на составляющей: проекции \textit{сверху} и проекции \textit{снизу} \citep{winkler1997focus}. В первом случае \citep{gussenhoven1983focus} утверждается, что признак фокуса присваивается наиболее высокому узлу и затем просачивается ``вниз'', пока не находит наиболее подходящее место для реализаци; во втором случае \citep{selkirk1984phonology, selkirk1996prosodic} признак фокуса изначально находится на составляющей с фразовым ударением, после чего проецируется вверх, маркируя сферу действия фокуса.

\subsection{Семантика фокуса}

В данном разделе мы рассмотрим существующие подходы к анализу семантики фокуса. В \ref{alternativeSemantics} мы изложим основные положения \textit{семантики альтернатив} \citep{rooth1985association,rooth1992theory}; в \ref{structuredMeanings} мы рассмотрим иной подход, названный теорией \textit{структурированных значений} \citep{Cresswell1985}, а также проведем его сравнение с семантикой альтернатив.

\subsubsection{Семантика альтернатив \citep{rooth1985association,rooth1992theory}} \label{alternativeSemantics}

Семантика альтернатив \citep{rooth1985association,rooth1992theory} является одним из наиболее известных и распространенных подходов к анализу семантики фокуса. Утверждается, что фокус привносит в процесс интерпретации высказывания \textit{множество альтернатив}, существующих параллельно с ``обычной'' семантикой высказывания.

\medskip

Этот подход близок к анализу семантики вопросов в работах \citep{hamblin1973questions,karttunen1977syntax}, где интерпретация вопроса \textit{What does Andrew smoke?} будет выглядеть как множество ответов $ \{ smoke(A, x) \ | \ object(x) \} $. В литературе разделились мнения относительно того, какого рода ответы должны входить в такое множество: должны ли это быть истинные и возможные \citep{hamblin1973questions}, истинные \citep{karttunen1977syntax} или просто возможные \citep{groenendijk1985semantics} ответы. В любом случае, множество $ \{ smoke(A, x) \ | \ object(x) \} $ --- то, что в \citep{rooth1985association} называется \textsc{фокусным значением} (англ. \textit{focus semantic value}) выражения $ Andrew\ smokes\ [\textsc{cigarettes}]_{F} $.

\medskip

Понятие фокусного значения используется для интерпретации чувствительных к фокусу элементов вроде английского \textit{only} или русского \textit{только}. Предположим, что $ S $ --- предложение, содержащее \textit{only} (например, $ Andrew\ only\ smokes\ [\textsc{cigarettes}]_{F} $), а $ M $ и $ R $ --- его фокусное и обычное значения соответственно. $ only(S) $ означает, что все альтернативы, кроме $ S $, не верны. Мы можем записать сказанное формально:

\begin{exe}
    \ex \label{simpleAlternatives} $ \forall P \in M, true(P) \rightarrow P = R $
\end{exe}

В таком случае для предложения выше мы будем иметь следующую интерпретацию:

\begin{exe}
    \ex \begin{xlist}
        \ex \textit{Every true proposition stating that Andrew smokes some object is the proposition that Andrew smokes cigarettes.}
        \ex $ \forall P \in \{ smoke(A, x) \ | \ object(x) \}, true(p) \rightarrow P = smoke(A, c) $
    \end{xlist}
\end{exe}

Однако приведенная в (\ref{simpleAlternatives}) интерпретация фокуса не является удовлетворительной в том смысле, что не учитывает \textit{контекст} высказывания, который может значительно ограничивать исходное множество всех возможных альтернатив. Например, очевидно, что в примере (\ref{restrictedAlternatives}) выбор происходит между двумя определенными альтернативами:

\begin{exe}
    \ex \begin{xlist} \label{restrictedAlternatives}
        \ex A: What does Andrew smoke: cigarettes or pipe?
        \ex B: He smokes only [\textsc{cigarettes}]\textsubscript{F}.
    \end{xlist}
\end{exe}

Для решения этой проблемы \citep{rooth1992theory} предлагает использовать понятие множества \textit{контекстно уместных} альтернатив, которое должно являться подмножеством фокусного значения. В таком случае значение $ only(S) $ можно записать следующим образом:

\begin{exe}
    \ex \label{salientAlternatives}
        $ \llbracket only(S) \rrbracket = \forall P \in ALT\ [true(P) \rightarrow (P = S')] $, где S' --- это обычное значение высказывания $ S $, а $ ALT $ --- множество контекстно уместных альтернатив, причем $ ALT $ является подмножеством фокусного значения $ S $.
\end{exe}

Для примера (\ref{restrictedAlternatives}) $ ALT $ будет множеством, состоящим из двух пропозиций: $ \{ Andrew\ smokes\ cigarettes,\ Andrew\ smokes\ pipe \} $.

\medskip

Проблемой семантики альтернатив в её форме, изложенной в \citep{rooth1985association}, является то, что с её помощью невозможно описать механизм ассоциации с фокусом в целом; нам необходимо прописывать правила ассоциации с фокусом для каждого типа чувствительной конструкции по отдельности, f а также напрямую использовать понятие фокусного значения в семантике отдельных лексических вхождений. \citep{rooth1992theory} предлагает называть такие теории интерпретации фокуса \textit{слабыми}. Например, в интерпретации для $ only(S) $ используется отсылка к $ ALT $, значение которого определяется \textit{фокусным} значением $ S $.

\medskip

\citep{rooth1992theory} разрабатывает собственную \textit{сильную} теорию интерпретации фокуса, которая не требует формулировки правил для отдельных конструкций. Перечислим типы конструкций, чувствительных к фокусу:

\begin{exe}
    \ex \begin{xlist} \label{focusConstructions}
        \ex \textbf{Фокусные наречия.} Если $ C $ --- область квантификации фокусного наречия с аргументом $ \alpha $ , то $ C \subseteq \llbracket \alpha \rrbracket^{F}. $ \\ Пример: \textit{only}.
        \ex \textbf{Контрастирующие выражения.} Если выражение $ \alpha $ контрастирует с выражением $ \beta $, то $ \llbracket \beta \rrbracket^{o} \in \llbracket \alpha \rrbracket^{F}. $ \\ Пример: [\textsc{Русский}]\textsubscript{F} \textit{солдат встречает} [\textsc{американского}]\textsubscript{F} \textit{солдата.}
        \ex \textbf{Выражения со скалярной импликацией.} Если $ C $ --- упорядоченное множество, использующееся для формирования импликатуры выражения $ \alpha $, то $ C \subseteq \llbracket \alpha \rrbracket^{F}. $ \\ Пример: \textit{Как прошел экзамен? --- Ну, я} [\textsc{сдал}]\textsubscript{F} (\textit{сдать на отлично}$ \implies $\textit{сдать}).
        \ex \textbf{Выражения типа ``вопрос--ответ''.} Для пары вопрос--ответ $ \langle \beta, \alpha \rangle $ верно, что $ \llbracket \beta \rrbracket^{o}  \subseteq \llbracket \alpha \rrbracket^{F}. $ \\ Пример: см. пример (\ref{pitchAccentVSfocus}).
    \end{xlist}
\end{exe}

Здесь $ \llbracket . \rrbracket^{o} $ означает ``обычное'' семантическое значение выражения.

\medskip

Далее замечается, что в каждой из приведенных конструкций речь идет о том, что некий семантический объект является либо элементом, либо подмножеством фокусного значения высказывания. Разница между этими случаями заключается в типе (в широком смысле) семантического объекта, причем перечисление возможных типов не должно входить в собственно теорию \textit{фокуса}.

\medskip

На основании этих рассуждений репертуар семантики альтернатив сокращается до единственного оператора $ \sim $ \citep{rooth1992theory}:

\begin{exe} 
    \ex \label{tildaOperator} Если $ \phi $ --- синтаксическое выражение, то $ \phi \sim C $ вводит пресуппозицию, что:
    \begin{xlist}
        \ex $ C \subseteq \llbracket \phi \rrbracket^{F} $, причем С содержит $ \llbracket \phi \rrbracket^{o} $ и еще как минимум один элемент, если $ C $ --- множество;
        \ex $ C \in \llbracket \phi \rrbracket^{F} $ и $ C \neq \phi $, если $ C $ --- индивидуальный элемент.
    \end{xlist}
\end{exe}

Преимуществом такого правила для интерпретации фокуса является тот факт, что мы не должны упоминать в нём конкретные конструкции, вступающие во взаимодействие с фокусом. Фокус просто вводит в интерпретацию \textit{некую} свободную переменную $ C $ и устанавливает ограничения на семантику (семантический тип) только \textit{этой переменной}. Более того, при задании семантического значения ``чувствительных к фокусу'' элементов мы можем в принципе обойтись без упоминания понятия фокуса; например, в \citep{rooth1992theory} для $ only(V\!P) $ предлагается следующая интерпретация:

\begin{exe}
    \ex \label{onlyVPredefined} $ \llbracket only\ V\!P \rrbracket = \lambda x.[\forall P [P \in C \land P(x) \rightarrow P = V\!P ]] $
\end{exe}

Здесь важно еще раз подчеркнуть, что $ C $ --- просто некая свободная переменная. Удаляя упоминания фокуса из семантического значения, мы также решаем проблему, связанную со случаями употребления $ only $ без эксплицитно выраженного фокуса (подробнее об этом см. раздел \ref{focusApproaches}).

\medskip

Приведем пример интерпретации следующего выражения с использованием определения из (\ref{tildaOperator}):

\begin{exe}
    \ex $ Mary\ only\ introduced\ [\textsc{Bill}]_{F}\ to\ Sue. $
\end{exe}

Дерево синтаксического разбора для этого примера будет выглядеть приблизительно так:

\begin{exe}
  \ex \label{tildaTree} \Tree [.S [.NP Mary ] [.VP  { $ only(C) $ } [.VP \qroof{$ introduced\ [Bill]_{F}\ to\ Sue $}.VP {$ \sim C $} ] ] ]
\end{exe}

Как видно из (\ref{tildaTree}), фокус интерпретируется на уровне VP. $ C $ здесь является свободной переменной и ведет себя как ``местоимение'', отсылающее ко множеству доступных альтернатив; после окончания деривации оно должно быть заполнено альтернативами вида $ \{ Introduce(x, \bm{Bill}, Sue),\\Introduce(x, \bm{Jack}, Sue), Introduce(x, \bm{Kyle}, Sue), ... \} $ (о фиксации этого множества см. ниже в (\ref{fixingC})).

\medskip

Под семантическим значением VP понимается обычное лямбда-выражение вида $ \lambda z.\lambda y.\lambda x.[Introduce(x, y, z)] $. Если подставить это выражение в (\ref{onlyVPredefined}), имеем следующую деривацию:

\begin{exe}
    \ex \label{} $ \lambda x_{1} .[\forall P [P \in C \land P(x_{1}) \rightarrow P = \lambda x_{2}.Introduce(x_{2}, Bill, Sue) ] $ \\ $ [\forall P [P \in C \land P(Mary) \rightarrow P = \lambda x_{2}.Introduce(x_{2}, Bill, Sue) ] $
\end{exe}

Как мы видим, результатом интерпретации на уровне семантики является выражение со свободной переменной $ C $. В \citep{rooth1992theory} предлагается считать, что окончательная фиксация значения этой переменной происходит на уровне прагматики. В примере (\ref{fixingCtext}) множество альтернатив дополнительно ограничено контекстом:

\begin{exe}
    \ex \label{fixingCtext} \begin{xlist}
        \ex $ Who\ did\ Mary\ introduce\ to\ Sue,\ Bill\ or\ John? $
        \ex $ Mary\ only\ introduced\ [\textsc{Bill}]_{F}\ to\ Sue. $
    \end{xlist}
\end{exe}

Результат интерпретации (\ref{fixingCtext}) на уровне прагматики будет выглядеть так: 

\begin{exe}
    \ex \label{fixingC} $ \exists C [ C = \{ Introduce(x, \bm{Bill}, Sue), Introduce(x, \bm{Jack}, Sue) \} \\ \land [\forall P [P \in C \land P(Mary) \rightarrow P = \lambda x_{2}.Introduce(x_{2}, Bill, Sue) ]] $
\end{exe}

\subsubsection{Структурированные значения \citep{klein1982intonation,krifka1992compositional}} \label{structuredMeanings}

Теория \textit{структурированных значений} \citep{klein1982intonation,krifka1992compositional} похожа на семантику альтернатив тем, что предполагает наличие двух компонент в интерпретации фокусного высказывания. Однако если в семантике альтернатив этими компонентами были стандартное и фокусное значение, то структурированным значением является кортежем из \textit{вопроса} (фона, англ. \textit{background}) и простого \textit{ответа} (фокуса):

\begin{exe}
    \ex \label{structuredIntro} \begin{xlist}
        \ex $ Kyle\ hates\ the\ [\textsc{fatboy}]_{F} $
        \ex $ \langle \lambda x.Hate(Kyle, x), \imath\ f\!atboy \rangle $
    \end{xlist}
\end{exe}

Обычное значение выражения получается путем функционального применения фона к фокусу. Если фокус находится на предикате $ Kyle\\~ [\textsc{hates}]_{F}\ the\ f\!atboy $, то значение должно выглядеть как

\begin{exe}
    \ex \label{structuredIntroPredicate} $ \langle \lambda P.[P(Kyle, \imath\ f\!atboy)], hate \rangle $
\end{exe}

В примерах (\ref{structuredIntro}, \ref{structuredIntroPredicate}) можно наблюдать не совсем типичную репрезентацию значения высказывания. Очевидно, что при классическом композициональном подходе невозможно добиться лямбда-абстракции внутреннего аргумента глагола при фиксированном внешнем аргументе (i.e., $ \lambda x.Hate(Kyle, x) $).

\medskip

В \citep{krifka1992compositional} для решения данной проблемы предлагается достаточно сложная альтернатива стандартному правилу функционального применения. Мы приведем (в целях ознакомления) определение семантических типов и формулировку \textit{расширенного} функционального применения, однако в дальнейшем будем использовать упрощенную нотацию, предлагаемую в \citep{beaver2008sense}.

\medskip


\begin{exe}
    \ex Семантические типы:
    \begin{xlist}
        \ex $ e $ есть тип сущностей, $ t $ --- тип истинностных значений.
        \ex Если $ \sigma, \tau $ --- типы, то:
        \begin{xlisti}
            \ex $ \sigma(\tau) $ есть тип функций от $ \sigma- $денотаций к $ \tau- $денотациям;
            \ex $ \sigma \LargeCdot \tau $ есть тип списка $ \sigma- $денотаций и $ \tau- $денотаций;
            \ex $ \langle \sigma, \tau \rangle $ есть тип структуры фон---фокус.
            
        \end{xlisti}
    \end{xlist}
\end{exe}


\begin{exe}
    \ex Расширенное функциональное применение ``(\ )'':
    \begin{xlist}
        \ex Если $ \alpha $ имеет тип $ \sigma(\tau) $ и $ \beta $ имеет тип $ \sigma $, то $ \alpha(\beta) $ имеет тип $ \tau $ и означает функциональное применение;
        \ex \textit{{\footnotesize Наследование фокуса от оператора:}}\\
        Если $ \langle \alpha, \beta \rangle $ имеет тип $ \langle (\sigma)(\tau)\mu, \sigma^{\LargerCdot} \rangle $ и $ \gamma $ имеет тип $ \tau $, то $ \langle \alpha, \beta \rangle(\gamma) $ имеет тип $ \langle (\sigma)\mu, \sigma^{\LargerCdot} \rangle $ и интерпретируется как $ \langle \lambda X_{\sigma}.[\alpha(X)(\gamma)], \beta \rangle $;
        \ex \textit{{\footnotesize Наследование фокуса от аргумента:}}\\
        Если $ \gamma $ имеет тип $ (\sigma)\tau $ и $ \langle \alpha, \beta \rangle $ имеет тип $ \langle (\mu)\sigma, \mu^{\LargerCdot} \rangle $, то $ \gamma(\langle \alpha, \beta \rangle) $ имеет тип $ \langle (\mu)\tau, \mu^{\LargerCdot} \rangle $;
        \ex \textit{{\footnotesize Наследование фокуса от оператора и аргумента:}}\\
        Если $ \langle \alpha, \beta \rangle $ имеет тип $ \langle (\sigma)(\tau)\mu, \sigma^{\LargerCdot} \rangle $ и $ \langle \gamma, \delta \rangle $ имеет тип $ \langle (\upsilon)\tau, \upsilon^{\LargerCdot} \rangle $, то $ \langle \alpha, \beta \rangle(\langle \gamma, \delta \rangle) $ имеет тип $ \langle (\sigma \LargeCdot \upsilon)\mu, \sigma^{\LargerCdot} \LargeCdot \upsilon^{\LargerCdot} \rangle $ и интерпретируется как $ \langle \lambda X_{\sigma} \LargeCdot Y_{\upsilon}.[\alpha(X)(\gamma(Y))], \beta \LargeCdot \delta \rangle $.
    \end{xlist}
\end{exe}

\medskip

Как было замечено выше, \citep{beaver2008sense} предлагают упрощенное представление деривации в рамках теории структурированных значений; мы будем придерживаться этой упрощенной нотации.

\medskip

Для случаев, когда выражение содержит несколько элементов, несущих фокус, требуется введение операций над \textit{составными объектами} \citep{krifka1992compositional}. Так, для интерпретации выражения $ Mary\ introduced\ [\textsc{Bill}]_{F}\\to\ [\textsc{Sue}]_{F} $ мы должны ввести операцию лямбда-абстракции относительно списков переменных:

\begin{exe}
    \ex $ \langle \lambda [x,y].Introduce(Mary, x, y), [Bill, Sue] \rangle $
\end{exe}

Также необходимо ввести правила для функционального применения выражения $ A $ к $ B $, если оба этих выражения являются кортежами \textit{фон, фокус}. Для этого введем операцию конкатенации списков:

\begin{exe}
    \ex $ [x_1, ..., x_i] \sqcap [y_1, ..., y_j] = [x_1, ..., x_i, y_1, ..., y_j] $
\end{exe}


Тогда для выражений $ A $ и $ B $, имеющих структурированные значения $ \langle \alpha_b, \alpha_f \rangle $ и $ \langle \beta_b, \beta_f \rangle $ соответственно, применение $ AB $ будет иметь следующий вид:

\begin{exe}
    \ex $ \langle \lambda[x, y].\alpha_b(x)(\beta_b(y)), \alpha_f \sqcap \beta_f \rangle $
\end{exe}

Для удобства нотации введем следующие обозначения. Пусть вырожденные случаи лямбда-абстракции и функционального применения не будут иметь никакого эффекта: $ \lambda \langle\ \rangle\!=\!\phi $ и $ \phi(\langle\ \rangle)\!=\!\phi $. Тогда будем считать, что ``$ \langle \phi $'' означает ``$ \langle \phi, [\ ] \rangle $'', т.е. структурированное значение с фоном, но без фокуса; также положим, что ``$ \phi \rangle $'' означает `` $ \langle \lambda X.X, \phi \rangle $'', т.е. фокус с тривиальным фоном (чисто фокусное значение). 

\medskip

Рассмотрим теперь деривацию\footnote{Еще раз отметим, что приведенная деривация является упрощенной записью достаточно сложного формализма, разработанного в \citep{krifka1992compositional}.} выражения $ Andrew\ gave\ [\textsc{Kyle}]_{F}\ cigarettes $ \citep[4.15]{beaver2008sense}:

\begin{exe}
    \ex \label{derivingStruct}
        \begin{flalign*}
            Kyle                                    &\mapsto \langle k &                                    \\
            [Kyle]_{F}                              &\mapsto k \rangle                                      \\
            gave                                    &\mapsto \langle Give                                   \\
            gave\ [Kyle]_{F}                        &\mapsto \langle Give, k \rangle                        \\
            cigarettes                              &\mapsto \langle c                                      \\
            gave\ [Kyle]_{F}\ cigarettes            &\mapsto \langle \lambda \beta. Give(\beta)(c), k \rangle       \\
            Andrew                                  &\mapsto \langle a                                      \\
            Andrew\ gave\ [Kyle]_{F}\ cigarettes    &\mapsto \langle \lambda x. Give(a)(x)(c), k \rangle    \\
        \end{flalign*}
\end{exe} 

Далее, для конфигурации $ \llbracket only(V\!P) \rrbracket $ предлагается следующая интерпретация:

\begin{exe}
    \ex $ \lambda[B,F].\langle \lambda x.\forall \gamma [B(\gamma)(x) \rightarrow \gamma = F ] $
\end{exe}

В таком случае мы имеем следующую интерпретацию выражения $ Andrew\ only\ gave\ [\textsc{Kyle}]_{F}\ cigarettes $:

\begin{exe}
    \ex \label{derivingStruct}
        \begin{align*}
            only                                          &\mapsto \lambda[B,F].\langle \lambda x.\forall \gamma [B(\gamma)(x) \rightarrow \gamma = F ] \\
            only\ gave\ [Kyle]_{F}\ cigarettes            &\mapsto \langle \lambda x.\forall \gamma [\lambda [\beta.Give(\beta)(c)](\gamma)(x) \rightarrow \gamma = b] \\
                                                          &\mapsto \langle \lambda x.\forall \gamma [Give(\gamma)(c)(x) \rightarrow \gamma = b] \\
            Andrew                                        &\mapsto \langle a                                      \\
            Andrew\ only\ gave \\ [Kyle]_{F}\ cigarettes  &\mapsto \langle \forall \gamma [Give(\gamma)(c)(a) \rightarrow \gamma = b ]   \\
        \end{align*}
\end{exe} 


\subsection{Прагматика фокуса}\label{focusPragmatics}

В данном разделе мы кратко рассмотрим подходы к анализу фокуса с точки зрения прагматики.

\subsubsection{О классификации подходов к анализу фокуса} \label{focusApproaches}

В разделе \ref{alternativeSemantics} было упомянуто, что \citep{rooth1992theory} различал \textit{слабые} и \textit{сильные} теории фокуса. Слабые теории \citep{rooth1985association} отличаются тем, что в них чувствительные к фокусу элементы напрямую обращаются к фокусному значению отдельных составляющих (\ref{weak}), в то время как сильные \citep{rooth1992theory} избегают этого (\ref{strong}). \citep{beaver2008sense} используют термины \textit{конвенциональная} и \textit{свободная} ассоциация с фокусом, в целом соответствующие слабой и сильной ассоциации в \citep{rooth1992theory}.

\begin{exe}
    \ex \begin{xlist}
        \ex \label{weak} $ \llbracket only\ V\!P \rrbracket = \lambda x.[\forall P [P \in \llbracket V\!P \rrbracket^{F} \land P(x) \rightarrow P = V\!P]] $
        \ex \label{strong} $ \llbracket only\ V\!P \rrbracket = \lambda x.[\forall P [P \in C \land P(x) \rightarrow P = V\!P ]] $
    \end{xlist}
\end{exe}

В (\ref{strong}) свободная переменная $ C $ чаще всего определяется набором фокусных альтернатив, вводящихся в пресуппозицию на этапе синтаксической деривации (см. (\ref{tildaTree})). Преимуществом данного подхода (помимо эстетических соображений) является то, что в сильной теории мы не вынуждены утверждать, что чувствительные к фокусу выражения могут употребляться только в контекстах с эксплицитным грамматическим фокусом; существование \textit{безакцентного} фокуса может служить аргументом в пользу сильных теорий:

\begin{exe}
    \ex A: Мне кажется, все знали, что Кламм общается только\footnote{Мы полагаем, что в релевантных для данного обсуждения аспектах русское \textit{только} ведет себя так же, как английское \textit{only}.} с [\textsc{Фридой}]\textsubscript{F}. \\
        B: Даже [\textsc{К.}]\textsubscript{F} знал, что Кламм общается только с [\textsc{Фридой}]\textsubscript{SOF}, но это его мало заботило.
\end{exe}

Второе упоминание Фриды в ответе \textit{В} не несет на себе эксплицитного фразового ударения, но употребление \textit{только} в этом контексте все равно является грамматичным; это явление называется вторичным фокусом (англ. \textit{second occurence focus}) \citep{partee199911}. В (\ref{preAccentlessOnly}) дан еще однин пример безакцентного употребления \textit{only} (здесь вторичный фокус предшествует фразовому ударению):

\begin{exe} 
    \ex \label{preAccentlessOnly} A: I hear that John only gave [\textsc{a book}]\textsubscript{F} to Mary. \\
    B: True, but John only gave [\textsc{a book}]\textsubscript{SOF} SOF to [\textsc{many people}]\textsubscript{F}.
\end{exe}

Стоит заметить также, что в \citep{beaver2008sense} нередко говорится о чисто прагматическом (англ. \textit{purely pragmatic}) подходе к анализу фокуса (так, например, говорится о подходе \citep{roberts1996information}, см. раздел \ref{robertsFocus}); можно считать, что здесь имеется в виду то же, что и в случае свободной (сильной) ассоциации. В этом смысле теория \citep{rooth1992theory} является не менее ``прагматической'', чем \citep{roberts1996information}, однако в последней гораздо больше внимания уделяется непосредственно механизму фиксации множества альтернатив в прагматике. По этой причине мы приводим изложение \citep{roberts1996information} именно в разделе о прагматике фокуса.

\subsubsection{Прагматический механизм ассоциации с фокусом \citep{roberts1996information,beaver2008sense}} \label{robertsFocus}

Подход \citep{roberts1996information} является чисто прагматической (сильной) теорией фокуса: все чувствительные к фокусу выражения ассоциируются свободно. Так, \textit{only} получает интерпретацию, в которой рестриктором для квантификации является свободная переменная (ср. с интерпретацией в \citep{rooth1992theory}):

\begin{exe}
    \ex $ Andrew\ only\ smokes\ [\textsc{cigarettes}]_{F}\ on\ Fridays. $
    \begin{xlist}
            \ex \label{robertsOnly} $ \forall P [P \in R \land true(P) \rightarrow P = smoke(Andrew, cigarettes)] $
            \ex $ R = \{smoke(Andrew, x)\ |\ x \in D \} $
    \end{xlist}
\end{exe}

В \citep{roberts1996information} подробно обсуждается механизм фиксации значения переменной $ R $ (множества контекстно доступных альтернатив). Предлагается формальное развитие идеи \citep{rooth1985association} о том, что это множество должно зависеть от \textit{обсуждаемого вопроса} (англ. \textit{question under discussion}).

\medskip

Прежде всего \citep{roberts1996information} дает собственное определение семантики вопроса. Вопрос интерпретируется как абстракция по каждому из содержащихся в нем \textit{wh}-слов: $ \llbracket W\!ho\ arived? \rrbracket = who(\lambda x.[x\ arrived]) $. Множество Q-альтернатив вопроса задается как применение каждого из абстрагированных \textit{wh}-слов ко всем доступным сущностям в модели.

\begin{exe}
    \ex \label{robertsQuestion} Множество \textbf{Q-альтернатив} вопроса:\\ $ Q \mathhyphen alt(\alpha) = \{ P\ |\ \exists u^{i-1}_{\in D}, ..., u^{i-n}_{\in D} [ P = \llbracket \beta \rrbracket (u^{i-1}) ... (u^{i-n}) ]]\} $,
    где $ \alpha $ имеет вид $ wh_{i-1}, ..., wh_{i-n}(\beta) $; $ \{ wh_{i-1}, ..., wh_{i-n} \} $ --- (возможно, пустое) множество \textit{wh}-слов в $ \alpha $, а $ D $ --- область определения модели (множество всех людей для \textit{who} и множество всех вещей для \textit{what}).
\end{exe}

Далее выделяются критерии, которым должно соответствовать высказывание с фокусом; этими критериями являются \textbf{релевантность} и \textbf{конгруэнтность}:

\begin{exe}
    \ex \begin{xlist}
            \ex Высказывание $ m $ \textbf{релевантно} для обсуждаемого вопроса  $ q $ в одном из двух случаев:
                \begin{xlisti}
                    \ex $ m $ является частичным ответом на вопрос $ q $ ($ m $ --- утверждение);
                    \ex $ m $ является частью стратегии ответа на вопрос $ q $ ($ m $ --- вопрос).
                \end{xlisti}
            \ex Ответ $ \beta $ \textbf{конгруэнтен} вопросу $\!? \alpha $, если его фокусное значение совпадает со множеством Q-альтернатив вопроса: $ \llbracket \beta \rrbracket^{F} = Q \mathhyphen alt(\alpha) $.
    \end{xlist}
\end{exe}

В таком случае значением $ R $ для (\ref{fuckedUpExpression}) будет являться множество Q-альтернатив для вопроса в (\ref{fuckedUpQuestion}):

\begin{exe}
    \ex \begin{xlist}
            \ex \label{fuckedUpQuestion} Какой предмет/предметы является таковым, что Андрей не имеет других свойств, кроме курения этого предмета/предметов по пятницам?\footnote{Для применения определения вопроса из (\ref{robertsQuestion}) предполагается соответствие английского \textit{which} русскому \textit{какой}.}
            \ex \label{fuckedUpExpression} По пятницам Андрей курит только [\textsc{сигареты}]\textsubscript{F}.
    \end{xlist}
\end{exe}

\citep{roberts1996information} замечает, что вопрос в (\ref{fuckedUpQuestion}), хотя и полностью соответствует определениям (\ref{robertsOnly}, \ref{robertsQuestion}), не может быть задан в исходной форме: при условии, что Андрей существует, помимо курения сигарет по пятницам у него должно быть хотя бы свойство самоидентичности.

\medskip

Утверждается, что вследствие максимы качества по Грайсу (англ. \textit{Gricean Maxim of Quality}) вопрос (\ref{fuckedUpQuestion}) может быть употреблен в дискурсе только в том случае, если подразумевается его релевантность для вопроса (\ref{robertsGeneralQuestion}):

\begin{exe} 
    \ex \label{robertsGeneralQuestion} Какой предмет/предметы является таковым, что Андрей имеет свойство курения этого предмета/предметов по пятницам? \\
    (Что курит Андрей по пятницам?)
\end{exe}

Иными словами, область определения (англ. \textit{domain}) \textit{only} сокращается до множества предикатов курения кем-то чего-то по пятницам. В таком случае (\ref{fuckedUpQuestion}) приобретает следующую трактовку:

\begin{exe}
    \ex \label{fuckedUpQuestionCorrected} Какой предмет/предметы является таковым, что Андрей не имеет других свойств курения предмета/предметов по пятницам, кроме курения этого предмета/предметов по пятницам?
\end{exe}

\citep{roberts1996information} утверждает, что поскольку любой полный ответ на (\ref{fuckedUpQuestionCorrected}) является также ответом на (\ref{robertsGeneralQuestion}), возможен дискурс в (\ref{robertsGeneralizedDiscourse}):

\begin{exe}
    \ex \label{robertsGeneralizedDiscourse} \begin{xlist}
            \ex \label{robertsGeneralizedDiscourseQuestion} Что курит Андрей по пятницам?
            \ex \label{robertsGeneralizedDiscourseAnswer} По пятницам Андрей курит только [\textsc{сигареты}]\textsubscript{F}.
    \end{xlist}
\end{exe}

\citep{kadmon2001formal} замечает, что подобное рассуждение не дает нам инструментов для точного задания множества альтернатив. Предположим, что в примере (\ref{robertsOnly}) $ R $ содержит не только предсказываемый \citep{roberts1996information} набор альтернатив, но также альтеративу, звучащую как \textit{Андрей курит сигареты по понедельникам}. В таком случае вопрос в (\ref{fuckedUpQuestionCorrected}) будет иметь следующий вид:

\begin{exe}
    \ex \label{robertsFuckedUpQuestion} Какой предмет/предметы является таковым, что либо Андрей не имеет других свойств курения предмета/предметов по пятницам, кроме курения этого предмета/предметов по пятницам, либо Андрей ничего не курит по пятницам и курит сигареты по понедельникам?
\end{exe}

Несмотря на то, что такой вопрос звучит очень странно, любой полный ответ на (\ref{robertsFuckedUpQuestion}) будет также являться ответом на (\ref{robertsGeneralizedDiscourseQuestion}), а значит для \citep{roberts1996information} значение $ R $, определенное таким образом, будет являться удовлетворительным.

\subsubsection{Анализ фокуса в фреймворке DRT \citep{geurts2004interpreting}}

\section{Фокус и фокусные частицы}

В данном разделе мы дадим общее описание семантики и прагматики фокусных частиц их классификацию (\ref{focusParticles}); мы также подробно рассмотрим класс \textit{скалярных} фокусных частиц (\ref{scalarOperators}).

\subsection{Фокусные частицы: обзор и терминология} \label{focusParticles}

Прежде чем начать обсуждение фокусных частиц, необходимо установить терминологию, которая будет в дальнейшем использоваться в данной работе. Мы предлагаем заменить термин ``фокусная частица'' (каковыми являются, например, русские \textit{даже} и \textit{аж}) более широким термином \textit{фокусный оператор}, который позволил бы в дальнейшем объединить под одним названием как упомянутую частицу \textit{даже}, так и, например, наречие \textit{только}.

\medskip

Под фокусным оператором мы будем понимать лексический элемент, чувствительный к фокусу и взаимодействующий со значением модифицируемой пропозиции. Употребление фокусного оператора в высказывании влияет либо на пресуппозицию, либо на истинностное значение высказывания. Примером фокусного оператора, вдлияющего на истинность высказывания, может служить англ. \textit{only} (\ref{truthValues}), который мы повторяем ниже как (\ref{truthValuesAgain}):

\begin{exe}
    \ex \begin{xlist} \label{truthValuesAgain}
        \ex \label{truthValues1} John only introduced [\textsc{Bill}]\textsubscript{F} to Sue.
        \ex \label{truthValues2} John only introduced Bill to [\textsc{Sue}]\textsubscript{F}.
    \end{xlist}
\end{exe}

В примере (\ref{truthValues1}) утверждается, что единственным человеком, которого представили Сью, был Билл. В (\ref{truthValues2}), напротив, говорится, что Билла представили только Сью (в то время как Сью могли также представить Джона, Гарри, etc.) 

\medskip

Фокусные операторы могут быть \textbf{скалярными} и \textbf{нескалярными}. Фокусный оператор является скалярным, если задает (строгий) порядок на множестве фокусных альтернатив. Например, говоря \textit{Дима подал руку даже} [\textsc{Марин ле Пен}]\textsubscript{F}, мы подразумеваем, что Дима может подать руку далеко не всем людям; можно сказать, что множество (контекстно релевантных) людей упорядочено в соответствии с тем, насколько верноятной является ситуация, в которой Дима подал им руку (например: \textit{Жан-Люк Меланшон} $ > $ \textit{Франсуа Фийон} $ > $ \textit{Марин Ле Пен}), и что элемент, модифицируемый \textit{даже}, должен занимать на такой шкале высокое положение.  Заметим, что подход, основанный на вероятностной интерпретации упорядочивания альтернатив, является распространенным, но не лишен проблем (см., например, раздел \ref{scalarOperators}).

\medskip

Помимо скалярности фокусные операторы могут обладать свойством \textbf{аддитивности}. Аддитивность требует, чтобы среди множества фокусных альтернатив была верна хотя бы одна пропозиция, кроме модифицируемой. В примере (\ref{additiveAndNot}) русское \textit{даже} является аддитивным оператором, а \textit{аж} --- нет:

\begin{exe}
    \ex \label{additiveAndNot} \begin{xlist}
        \ex На приём пришел \textit{даже} президент (\# но больше никто не пришел).
        \ex На приём пришел \textit{аж} президент (но больше никто не пришел).
    \end{xlist}
\end{exe}

\subsection{Классификация фокусных операторов} \label{scalarOperators}

В данной работе мы с некоторыми дополнениями принимаем классификацию фокусных операторов, разработанную в \citep{gast2011scalar}. \citep{gast2011scalar} предлагают типологию скалярных аддитивных операторов, основанную на соотношении модифицируемой пропозиции со множеством контекстных альтернатив. Несмотря на то, что в цитируемой работе речь идет именно о типологии скалярных \textit{аддитивных} операторов, аддитивность (в отличие от скалярности) нигде не фигурирует в ней как дифференциальный признак\footnote{В разделе \ref{additivityClassification} мы предложим классификацию фокусных частиц, учитывающую свойство аддитивности.}: так, в типологии \citep{gast2011scalar} легко определить место для русского (нескалярного) \textit{аж} (см. \ref{auweraTypology} и аргументацию в разделе \ref{azh}).

\medskip

Как было замечено выше, \citep{gast2011scalar} принимают точку зрения \citep{rooth1985association} о семантике фокуса как о множестве контекстно заданных альтернатив.  Скалярные фокусные частицы предлагается считать пропозициональными операторами, вводящими пресуппозицию о том, что модифицируемая пропозиция \textit{прагматически сильнее} контекстных альтернатив.

\medskip

Понятие прагматической силы определяется следующим образом:

\begin{exe}
    \ex \label{pragmaticStrength} Пропозиция $ x $ \textbf{прагматически сильнее} пропозиции $ y $ (в отношении обсуждаемого вопроса $ Q $), если релевантные контекстные импликации $ y $ (в отношении $ Q $) являются логическим следствием релевантных контекстных импликаций $ x $.
\end{exe}

\textit{Контекстные импликации} --- это утверждения, выводимые одновременно из высказывания и контекста, но из высказывания или контекста по отдельности \citep{sperber2004experimental}; контекстные импликации релевантны в отношении обсуждаемого вопроса, если они дают некоторую информацию о предмете этого вопроса (либо уточняют его). 

\medskip

\citep{gast2011scalar} используют понятие контекстных импликаций как возможное решение проблемы стандартного (вероятностного) анализа семантики скалярных операторов. Проблема заключается в том, что не во всех случаях можно утверждать, что пропозиция, модифицируемая скалярным оператором, \textit{менее вероятна} своих альтернатив (\ref{failedProbabilityAnswer}):

\begin{exe}
    \ex \label{introLeatherSeats} \begin{xlist}
    \ex \label{failedProbabilityQUD} \textit{(Обсуждаемый вопрос)} Насколько темными являются кожаные автомобильные сидения?
    \ex \label{failedProbabilityAnswer} По эстетическим соображения кожаные сидения в автомобилях чаще всего окрашиваются в темно-серый цвет, в основном даже в черный.
    \end{xlist}
\end{exe}

В примере (\ref{failedProbabilityAnswer}) невозможно говорить о том, что черная окраска сидений является менее вероятной алтернативой. 

\medskip

Ниже мы рассмотрим пример использования понятия контекстных импликаций в случае обсуждаемого вопроса из (\ref{introLeatherSeats}):

\begin{exe}
    \ex \label{developedLeatherSeats} \textit{(Обсуждаемый вопрос)} Насколько темными являются кожаные автомобильные сидения? \begin{align*}
    \text{\textsc{Пропозиция}}  \hspace{2.23cm}                  &~ \hspace{0.16cm} \text{\textsc{Релевантная импликация}} \\
    \text{Сидение черное}       \hspace{1.12cm}  \longrightarrow &~ \hspace{0.16cm} \text{Сидение полностью темное}        \\
    \text{Сидение темно-серое}  \hspace{0.16cm}  \longrightarrow &~ \hspace{0.16cm} \text{Сидение очень темное} 
    \end{align*}
\end{exe}

В (\ref{developedLeatherSeats}) пропозиция \textit{Сидение темно-серое} не является логическим следствием пропозиции \textit{Сидение черное}, однако их контекстные импликации находятся в отношениях логического следования.

\medskip

\citep{gast2011scalar} утверждают, что множество альтернатив для фокусных операторов упорядочено именно по признаку логической обусловленности их контекстных импликаций, а не не по признаку вероятности. \citep{gast2011scalar} предлагают типологию фокусных операторов, построенную на соотношении прагматической силы (\ref{pragmaticStrength}) модифицируемой пропозиции и множества фокусных альтернатив.

\medskip

В построении типологии они отталкиваются от проблемы неоднозначности класса примеров с оператором $ even $:

\begin{exe}
    \ex \label{ambigEven} \begin{xlist}
    \ex \label{ambigEvenMurder} \textit{(Контекст убийства)} \\
    --- Bill is accused of murder, but I’m sure he’s innocent. \\
    --- I refuse to believe that Bill even $ [\textsc{slapped}]_F $ that man.
    \ex \label{ambigEvenInsult} \textit{(Контекст оскорбления)} \\
    --- Yesterday Bill insulted and slapped his collegue! \\
    --- This is so not typical of him. I refuse to believe that Bill even $ [\textsc{slapped}]_F $ that man.
    \end{xlist}
\end{exe}

В обоих примерах из (\ref{ambigEven}) фокусный оператор находится в \textit{обратно упорядоченном}\footnote{Данный контекст также можно считать контекстом с убывающей монотонностью (англ. \textit{downward entailing}) или контекстом с отрицательной полярностью (англ. \textit{negative polarity}); эти контексты различиются, однако различия между ними нерелевантны для нашего обсуждения.} (англ. \textit{scale-reversing}) контексте. В таких контекстах $ even $  модифицирует пропозиции, в обычном случае являющиеся прагматически слабыми: так, в (\ref{ambigEvenMurder}) $ slapped\ that\ man $ прагматически слабее фокусных альтернатив (конекстные импликации удара являются логически более слабыми, чем импликации увечья или убийства). Однако в (\ref{ambigEvenInsult}), в контексте оскорбления, $ even $ модифицирует прагматически сильную пропозицию.

\medskip

Существует два подхода к анализу данной проблемы: синтаксический и лексикалистский. В синтаксическом подходе предлагается считать, что в контексте убийства (\ref{ambigEvenMurder}) $ even $ интерпретируется \textit{in situ} и имеет сферу действия, в которую входит только сентенциальный аргумент ($[Bill\ even\ slapped\ that\ man] $); именно эта пропозиция является базой для фокусных альтернатив. Для (\ref{ambigEvenInsult}) предлагается считать, что $ even $ передвигается в синтаксически более высокую позицию; в таком случае в сфере действия оказывается высказывание целиком, и тогда $ even $ снова модифицирует прагматически сильную пропозицию.

\medskip

Последнее высказывание станет более очевидным, если рассмотреть контекстные альтернативы примеров из (\ref{ambigEven}) в сочетании с предполагаемыми обсуждаемыми вопросами:

\begin{exe}
    \ex \label{questionHarm} \textit{(Обсуждаемый вопрос: степень урона)} \\
        --- To what extent did Bill harm that man? \\
        --- I refuse to believe that $ [_{Scope}$ Bill even $ [\textsc{slapped}]_F $ that man $ ] $. \\
        $ Bill\ (killed\ >_{p}\ slapped >_{p}\ insulted)\ that\ man. $
\end{exe}

\begin{exe}
    \ex \label{questionInnocence} \textit{(Обсуждаемый вопрос: уверенность в невиновности)} \\
    --- To what extent are you convinced of Bill’s innocence? \\
    --- $ [_{Scope}$ I refuse to believe that Bill even $ [\textsc{slapped}]_F $ that man $ ] $. \\
         $ I\ re\!f\!use\ to\ believe\ that\ Bill\ (insulted\ >_{p}\ slapped >_{p}\ killed)\ that\ man. $
\end{exe}

В (\ref{questionInnocence}) из-за широкой сферы действия $ even $ обсуждаемым вопросом является то, насколько говорящий убежден в невиновности Билла. Уверенность в том, что Билл никого не убивал (но готовность допустить, что Билл ударил или оскорбил кого-то), является прагматически более слабой, чем уверенность в том, что Билл даже никого не оскорблял.

\medskip

Стоит заметить, чото проблемой синтаксического подхода к анализу неоднозначности $ even $ является невозможность объяснить, чем мотивировано передвижение, обеспечивающее широкое прочтение (контекст оскорбления): $ even $ может легко интерпретироваться $ in\ situ $.

\medskip

Лексикалистский подход заключается в постулировании наличия двух (омонимичных) лексических вхождений \citep{rooth1985association,giannakidou2007landscape}: $ \textsc{Even}_{neg} $ (имеет отрицательную полярность) и $ \textsc{Even}_{neut} $ (не имеет полярности). $ \textsc{Even}_{neut} $ интерпретируется обычным образом (модифицирует прагматически сильные пропозиции); $ \textsc{Even}_{neg} $ модифицирует прагматически \textit{слабые} пропозиции и может употребляться только в обратно упорядоченных контекстах. В таком случае мы можем объяснить неоднозначность (\ref{ambigEven}) напрямую: в контексте убийства мы видим употребление $ \textsc{Even}_{neg} $, а в контексте оскорбления ---  $ \textsc{Even}_{neut} $.

\medskip

В пользу данного анализа свидетельствуют типологические данные: в некоторых языках (например, в немецком) в лексиконе предусматриваются неомонимичные вхождения для  $ \textsc{Even}_{neut} $ и  $ \textsc{Even}_{neg} $. \citep{gast2011scalar} вводят понятия \textit{высоких} (англ. \textit{beyond}) и  \textit{низких} (англ. \textit{beneath}) операторов:

\begin{exe}
    \ex \textbf{Высокие операторы} --- скалярные аддитивные операторы, локальная пропозиция которых прагматически сильнее контекстных альтернатив; они могут употребляться как в обычных, так и в обратно упорядоченных контекстах.
\end{exe}

\begin{exe}
    \ex \textbf{Низкие операторы} --- скалярные аддитивные операторы, локальная пропозиция которых прагматически сильнее контекстных альтернатив; они могут только в обратно упорядоченных контекстах.\footnote{ \citep{gast2011scalar} замечают, что в случае, если низкий оператор употребляется во вложенной клаузе, необходимо, чтобы пропозиция матричной клаузы была прагматически сильной. Это демонстрируется на следующем примере:
        \begin{exe}
            \ex \begin{xlist}
                \ex Every student that even handed in $ [one]_F $ assignment got an A.
                \ex \# Every student that even handed in $ [one]_F $ assignment was wearing blue jeans.
            \end{xlist}
        \end{exe}}
\end{exe}

Среди низких опрераторов \citep{gast2011scalar} различают \textit{отрицательные} операторы (\ref{negative}), которые могут употребляться только в контексте сентенциального отрицания, и \textit{неотрицательные} операторы (\ref{nonnegative}), которые могут употребляться в любых других обратно упорядоченных контекстах:

\begin{exe}
    \ex \label{negative} {\footnotesize Немецкий} \begin{xlist}
        \ex \label{pitchAccentF}
            \gll Nicht einmal [seine Hunde]\textsubscript{F} gehorchen ihm. \\
                 not even his dogs obey him. \\
            \glt `Not even his dogs obey him.'
        \ex \label{pitchAccentF}
            \gll  *Es ist undenkbar, dass ihm einmal [seine Hunde]\textsubscript{F} gehorchen. \\
                  *it is unthinkable that him even his dogs obey \\
            \glt  `It is inconceivable that even his dogs obey him.'
    \end{xlist}
\end{exe}

\begin{exe}
    \ex \label{nonnegative} {\footnotesize Венгерский \citep[ex. 4]{abrusan2007even}} \begin{xlist}
        \ex \label{pitchAccentF} {\footnotesize\textit{Употребление akár невозможно при сентенциальном отрицании}}
            \gll *Péter nem üdvözölheti akár Marit sem. \\
                 *Peter not greet.can even Mari either \\
            \glt `Peter may not even greet Mari.'
        \ex \label{pitchAccentF} {\footnotesize\textit{Употребление akár возможно при высоком отрицании}}
            \gll  Nem igaz, hogy Péter akár egy példát is megoldott. \\
                  not true that Peter even one exercise too solved \\
            \glt  `It is not true that Peter solved even one exercise.'
        \ex \label{pitchAccentF} {\footnotesize\textit{Употребление akár возможно в условном предложении}}
            \gll  Ha akár egy ember is megszólal, kiüríttetem a termet. \\
                  if even one person too speaks empty.1SG the room \\
            \glt  `If even a single person says a word, I will empty the room.'
    \end{xlist}
\end{exe}

Типология фокусных операторов \citep{gast2011scalar} суммирована в (\ref{auweraTypology}):

\begin{table}[H]
\small
    \begin{exe}
    \ex \label{auweraTypology} 
        \Tree [.{\textsc{Аддитивные операторы}} [.{\textsc{Нескалярные}} ] [.{\textsc{Обобщенные}} [.{\textsc{Высокие}} ] [.{\textsc{Низкие}} [.{\textsc{Отрицательные}} ] [.{\textsc{Неотрицательные}} ] ] ] ]
    \end{exe}
\end{table}

Для русского языка распределение лексических вхождений выглядит следующим образом:

\begin{table}[H]
    \small
    \begin{tabular}{C{2.5cm}|C{2.5cm}|C{2.5cm}|C{2.5cm}}
    \multicolumn{4}{c}{\textsc{Аддитивные операторы}}                   \\ \hline
    \textsc{Нескалярные} & \multicolumn{3}{c}{\textsc{Обобщенные}}      \\ \cline{2-4}
    ~ & \textsc{Высокие} & \multicolumn{2}{c}{\textsc{Низкие}}          \\ \cline{3-4}
    ~ & ~ & \textsc{Отрицательные} & \textsc{Неотрицательные}           \\
    \end{tabular}
\end{table}

\begin{table}[H]
    \begin{tabular}{|C{2.5cm} C{2.5cm}|C{2.5cm}|C{2.5cm}|} \hline
    \rowcolor{gray!20} \multicolumn{4}{|c|}{и}                                  \\ \hline
    \rowcolor{gray!20} \multicolumn{1}{|c|}{тоже} & \multicolumn{3}{c|}{даже}   \\ \hline
    ~ & ~ & \cellcolor{gray!20} ни & \cellcolor{gray!20} хоть (аж)               \\ \hline
    \end{tabular}
\end{table}



\section{Скалярные фокусные операторы в русском языке}

\subsection{Аддитивность как дифференциальный признак в классификации фокусных операторов} \label{additivityClassification}

\subsection{План анализа русских скалярных операторов} \label{additivityClassification}

В данном разделе мы рассмотрим семантику и дистрибуцию русских фокусных операторов \textit{даже}, \textit{аж}, \textit{хотя бы} и \textit{хоть}. При анализе каждого из интересующих нас операторов мы будем придерживаться общего подхода, который подразумевает формальное определение значения оператора и описание его поведения в различных синтаксических контекстах.

\subsection{Даже}

\subsubsection{Семантика русского \textit{даже}}

Наиболее близким эквивалентом русской фокусной частицы \textit{даже} является английское \textit{even}, семантике которого посвящено значительное количество исследований \citep{horn1969presuppositional,stalnaker1974pragmatic,rullmann1997even,iatridou2016our,kay1990even}. Поскольку в наиболее важных аспектах значение \textit{даже} идентично значению \textit{even}, при его описании мы будем давать ссылки на релевантную литературу по \textit{even}\footnote{При таких ссылках мы будем избегать повторения формулировки ``как и в случае англ. \textit{even}, для \textit{даже} верно, что...''}.

\medskip

Формально \textit{даже} может пониматься как двухместный оператор с первым аргументом типа $ \langle \sigma \rangle $ и вторым аргументом типа $ \langle \sigma \tau \rangle $ \citep{wagner2006association}:

\begin{exe}
    \ex $ \textit{даже}(x)(y) $, где $ x $ --- составляющая в фокусе, а $ \llbracket xy \rrbracket $ --- пропозиция: \begin{xlist}
        \ex \textit{Ассерция:} $ \llbracket xy \rrbracket $;
        \ex \textit{Пресуппозиция:} Для любой из релевантных альтернатив $ x' $, $ \llbracket xy \rrbracket $ менее вероятно, чем $ \llbracket x'y \rrbracket $;
        \ex \textit{Пресуппозиция:} Существует такая альтернатива $ x' $, что $ \llbracket x'y \rrbracket $ верно.
    \end{xlist}
\end{exe}

\begin{exe}
    \ex \label{onlyAssPres} Дима встретился даже с $ [\textsc{президентом}]_F $. \begin{xlist}
        \ex \textit{Ассерция:} Дима встретился с президентом.
        \ex \textit{Пресуппозиция (вероятости):} Президент --- человек, вероятность встретить которого является наименьшей.
        \ex \textit{Пресуппозиция (аддитивности):} Помимо президента Джон встретил еще хотя бы одного человека.
    \end{xlist}
\end{exe}

Для примера (\ref{onlyAssPres}) результат подстановки аргументов $ even $ будет выглядеть так: $ even(the\ President, \lambda x.Meet(Dima, x)) $.

\medskip

Природа пресуппозиции вероятности, как уже было замечено в разделе \ref{focusParticles}, до сих пор является предметом обсуждения; для простоты здесь и далее мы будем обращаться к ``классической'' вероятностной интерпретации в случаях, когда точный механизм упорядочивания шкалы альтернатив не является принципиально важным для обсуждения. 

\medskip

Факт \textit{обязательного} наличия пресуппозиция аддитивности у английского \textit{even} является спорным. В некоторых случаях аддитивность действительно имеет место:

\begin{exe}
    \ex Even $ [\textsc{the President}]_F $ came to the party. \# Nobody else did.
\end{exe}


Однако в ряде случаев это не так. \citep{wagner2015additivity} замечает, что \textit{even}, модифицирующее именные группы (NP-even), обязательно обладает свойством аддитивности, в то время как для \textit{even}, модифицирующего глагольные группы (VP-even), это неверно:

\begin{exe}
    \ex \textit{John was a favorite in the marathon. Did he win a medal?} \begin{xlist}
        \ex Oh yes. He won even the $ [\textsc{gold}]_F $ medal. (Означает, что Джон выиграл более одной медали.)
        \ex Oh yes. It’s even the case that John won the $ [\textsc{gold}]_F $ medal.
    \end{xlist}
\end{exe}

Для нас может быть интересен вопрос о том, как соотносится аддитивность русского \textit{даже} и английского \textit{even}: всегда ли аддитивно \textit{даже}, и если нет, то в каких случаях не проявляется его аддитивность. Приведем сначала примеры, где аддитивность даже очевидна:

\begin{exe}
    \ex \textit{Я слышал, что результаты университетского марафона были неожиданными.} \\
        \# Даже $ [\textsc{Лёша}]_F $ выиграл золотую медаль. (В марафоне должен быть только один победитель.)
\end{exe}

\begin{exe}
    \ex \textit{Лёша считался фаворитом на университетских марафоне (разыгрывался один комплект медалей). Он взял какую-нибудь медаль?} \\
        \# Он выиграл даже $ [\textsc{золотую медаль}]_F $. (Означает, что Лёша выиграл более одной медали, что является навозможным.)
\end{exe}

\begin{exe}
    \ex \textit{На прошлой неделе мы сделали всего лишь половину работы.} \\
        \# На этой неделе мы не сделали даже $ [\textsc{ничего}]_F $.
\end{exe}

\begin{exe}
    \ex \# Даже $ [\textsc{Россия}]_F $ является самой большой страной в мире.
\end{exe}

Тем не менее, существует ряд контекстов, где аддитивное прочтение невозможно:

\begin{exe}
    \ex A: Петя младший научный сотрудник? \\
        В: Петя даже $ [\textsc{старший}]_F $ научный сотрудник.
\end{exe}

\begin{exe}
    \ex \textit{Дима большой любитель потанцевать, но вчера он был очень грустный и почти ни с кем не танцевал.} \\
        Дима даже танцевал только с $ [\textsc{Машей}]_F $.
\end{exe}

Вне зависимости от конкретных причин, по которым поведение \textit{NP-даже} отличается от \textit{VP-даже} (см. \citep{wagner2015additivity}), мы можем утверждать, что пресуппозиция аддитивности для \textit{даже} действительно зависит от структурной позиции.

\subsubsection{\textit{Даже} и передвижение в LF}

\citep{vermeulen2011interpreting} замечает, что в английском языке DP, модифицированная скалярным оператором, может совершать передвижение в LF в более высокую позицию, что приводит к неоднозначности:

\begin{exe}
    \ex I knew that John had learnt even $ [\textsc{Spanish}]_F $. \begin{xlist}
        \ex \textit{Узкое прочтение:} Изучение испанского языка маловероятно (для Джона).
        \ex \textit{Широкое прочтение:} Осведомленность об изучении Джоном испанского языка маловероятна. 
    \end{xlist}
\end{exe}


\begin{exe}
    \ex I knew that John had learnt only $ [\textsc{Spanish}]_F $. \begin{xlist}
        \ex \textit{Узкое прочтение:} Джон выучил только испанский.
        \ex \textit{Широкое прочтение:} Я знал только о том, что Джон выучил испанский. 
    \end{xlist}
\end{exe}

\begin{exe}
    \ex I knew that John had learnt also $ [\textsc{Spanish}]_F $. \begin{xlist}
        \ex \textit{Узкое прочтение:} Джон выучил не только испанский.
        \ex \textsuperscript{?}\textit{Широкое прочтение:}\footnote{В данном случае утверждение о наличии широкого прочтения является спорным, поскольку его сложно отличить от узкого.} Я знал не только о том, что Джон выучил испанский. 
    \end{xlist}
\end{exe}

\begin{exe}
    \ex I knew that John had learnt at least $ [\textsc{Spanish}]_F $. \begin{xlist}
        \ex \textit{Узкое прочтение:} Джон выучил как минимум испанский.
        \ex \textit{Широкое прочтение:} Я, как минимум, знал, что Джон выучил испанский. 
    \end{xlist}
\end{exe}

Для русского \textit{даже} (и в большей степени для \textit{аж} и \textit{хотя бы/хоть}) такое передвижение невозможно:

\begin{exe}
    \ex Я знал, что Дима выучил даже $ [\textsc{иврит}]_F $. \begin{xlist}
        \ex \textit{Узкое прочтение:} Изучение иврита маловероятно (для Димы).
        \ex \# \textit{Широкое прочтение:} Осведомленность об изучении Димой иврита маловероятна. 
    \end{xlist}
\end{exe}

\begin{exe}
    \ex Я знал, что Дима выучил только $ [\textsc{иврит}]_F $. \begin{xlist}
        \ex \textit{Узкое прочтение:} Дима выучил только иврит.
        \ex \# \textit{Широкое прочтение:} Я знал только о том, что Джон выучил испанский. 
    \end{xlist}
\end{exe}

\subsection{Аж} \label{azh}

\subsubsection{Семантика русского \textit{аж}}

Русское \textit{аж} практически не обсуждалось в формальной лингвистической литературе; единственным формальным анализом можно считать работу \citep{tomaszevicz2011az}, где на материале польского, чешского, словацкого и русского языков рассматривается ``обобщенная''  семантика скалярного оператора, в русском языке существующего в лексиконе под видом \textit{аж}.

\medskip

\citep{tomaszevicz2011az} предлагает считать, что семантика \textit{аж} представляет собой (скалярную) противоположность семантики английского \textit{only/merely}, которому дается следующая семантика:

\begin{exe}
    \ex \label{tomaszOnly} \textsc{only/merely(P)} \\ {\footnotesize Пусть $ P $ --- пропозиция, $ S $ --- упорядоченное множество контекстных альтернатив (шкала), $ \sigma $ --- контекстуальный стандарт, при  этом $ a \gg_{\sigma} b $ означает, что $ a $ находится значительно выше $ b $ на шкале $ S $. Тогда:} \begin{xlist}
        \ex \textit{Ассерция:} $ \lnot \exists P'\ [true(P) \land P' \neq P \land P' >_{\sigma} P] $;
        \ex \label{tomaszOnlySecond} \textit{Пресуппозиция:} $ \sigma \gg_{\sigma} P $;
        \ex \label{tomaszOnlyFirst} \textit{Пресуппозиция:} Верно, что как минимум $ P $.
    \end{xlist}
\end{exe}

Пресуппозиция (\ref{tomaszOnlySecond}) ожидаемо утверждает, что модифицируемая пропозиция занимает достаточно низкое положение на шкале альтернатив.  В (\ref{tomaszOnlyFirst}) утверждается, что рассматриваемые альтернативы занимают на шкале как минимум то же положение, что и модифицируемая пропозиция. Этот компонент значения \textit{only} можно наиболее отчетливо наблюдать под отрицанием:

\begin{exe}
    \ex Maria is \textit{not merely} a manager ($ \rightarrow $ Maria is more than a manager).
\end{exe}

Основываясь на приведенном анализе, \citep{tomaszevicz2011az} описывает семантику \textit{аж} как прямую противоположность \textit{only/merely}:

\begin{exe}
    \ex \textsc{аж(P)} \\ {\footnotesize Пусть $ P $ --- пропозиция, $ S $ --- упорядоченное множество контекстных альтернатив (шкала), $ \sigma $ --- контекстуальный стандарт, при  этом $ a \gg_{\sigma} b $ означает, что $ a $ находится значительно выше $ b $ на шкале $ S $. Тогда:} \begin{xlist}
        \ex \label{tomaszAzAssert} \textit{Ассерция:} $ \lnot \exists P'\ [ true(P) \land P' \neq P \land P' <_{\sigma} P] $;
        \ex \textit{Пресуппозиция:} $ \sigma \ll_{\sigma} P $;
        \ex \label{tomaszAzPresupMax} \textit{Пресуппозиция:} Верно, что как максимум $ P $.
    \end{xlist}
\end{exe}

Данный анализ имеет ряд недостатков. Прежде всего, (\ref{tomaszAzAssert}) утверждает, что употребление \textit{аж} исключает возможность того, что альтернативы, располагающиеся на шкале ниже модифицируемой пропозиции, могут оказаться верны (эксклюзивность). Однако примеры из (\ref{azNonAdditive}) противоречат данному утверждению:

 \begin{exe}
    \ex \label{azNonAdditive} \begin{xlist}
        \ex На заседание пришел аж $ [ $\textsc{президент}$ ]_F $! Премьер-министр тоже пришел, но на него никто не обратил внимание. \\
            \textit{(Президент} $ >_{\text{статус}} $ \textit{Премьер-министр)}
        \ex Таня поехала путешествовать по России и доехала аж до Улан-Удэ! По пути она также побывала в Балкарии. \\
        \textit{(Улан-Удэ} $ >_{\text{расстояние}} $ \textit{Балкария)}
    \end{xlist}
\end{exe}

Кроме того, в (\ref{tomaszAzPresupMax}) утверждается, что рассматриваемые альтернативы не должны находиться на шкале выше, чем модифицируемая пропозиция.

\begin{exe}
    \ex \begin{xlist}
        \ex Таню приняли аж в $[$\textsc{NYU}$]_F$! И в MIT её тоже приняли!\\
        \textit{(MIT} $ >_{\text{престиж}} $ \textit{NYU)}
        \ex Это правда, что Таня поступила аж в $[$\textsc{MIT}$]_F$?
    \end{xlist}
\end{exe}

\subsubsection{\textit{Aж} и отрицание}

В примерах (\ref{azhPPI1}-\ref{azhPPI2}) демонстрируется неспособность \textit{аж} находиться в сфере действия сентенциального отрицания:

\begin{exe}
    \ex \label{azhPPI1} \begin{xlist}
        \ex Таня сфотографировалась аж с Ирен Хайм.\footnote{Некоторые носители считают (\ref{azhPPI1}a-\ref{azhPPI2}a) ограниченно коректными: вследствие разговорного оттенка, присущего частице \textit{аж}, эти примеры можно назвать ``неизящными'' (непривычными для письменного языка). Однако в сравнении с (\ref{azhPPI1}b-\ref{azhPPI2}b) все носители признают эти примеры допустимыми. Кроме того, (\ref{azhPPI2}a) был взят из выдачи поискового движка Google, т.е. не является искусственно сконструированным примером.}
        \ex \# Таня не сфотографировалась аж с Ирен Хайм.
    \end{xlist}
\end{exe}

\begin{exe}
    \ex \label{azhPPI2} \begin{xlist}
        \ex Вася сидел за одним столом аж с президентом Бразилии.
        \ex \# Вася не сидел за одним столом аж с президентом Бразилии.\footnote{Недопустимые выражения из примеров (\ref{azhPPI1}-\ref{azhPPI2}), очевидно, возможны в качестве корректирующего ответа на ложное утверждение (отрицание с повтором).}
    \end{xlist}
\end{exe}

\medskip

На первый взгляд можно утверждать, что \textit{аж} является элементом с положительной полярностью (англ. \textit{PPI}); особенности дистрибуции \textit{аж} подтверждают эту гипотезу:

\begin{exe}
    \ex \begin{xlist}
        \ex \label{azhMatrixNeg} Я не говорил, что Таня курила аж в кабинете директора.
        \ex \label{azhAlwaysNeg} Таня не всегда курила аж в кабинете директора.
    \end{xlist}
\end{exe}

В (\ref{azhMatrixNeg}) \textit{аж} без особых проблем существует при отрицании в матричной клаузе, а в (\ref{azhAlwaysNeg}) оно может употребляться ``под защитой'' кванторного наречия \textit{всегда}; все приведенные контексты входят в (считающуюся типичной) дистрибуцию PPI.

\medskip

Однако стоит выделить особый тип примеров, в которых \textit{аж} может существовать под отрицанием:

\begin{exe}
    \ex \label{azhNegativeEvents} \begin{xlist}
        \ex Вася не спал аж $ [\textsc{30 часов}]_F $.
        \ex \textit{Васе свойственно опаздывать всюду, иногда это приводит к последствиям.} \\ Например, когда Вася не успел аж $ [\textsc{на совещание} $ \\ $ \textsc{директоров}]_F $.
        \ex \textit{Вася ходил на официальные приёмы и демонстративно выказывал презрение высокопоставленным чиновникам (из-за своих политических убеждений). Презрение Вася выказывал посредством отказа совершить рукопожатие перед объективами камер. В понедельник он проигнорировал министра обороны, в среду --- премьер-министра.} \\
        Вчера Вася не пожал руку аж $ [\textsc{президенту}]_F $.
    \end{xlist}
\end{exe}

Исходя из соображений здравого смысла, мы не видим причин постулировать в примерах выше скрытое передвижение \textit{аж} в позицию выше отрицания; в то же время отказываться от PPI-статуса \textit{аж} представляется нежелательным.\footnote{Делать мы этого, конечно же, не будем.}

\medskip

Можно предположить, что в высказываниях из (\ref{azhNegativeEvents}) мы видим не сентанциальное отрицание, а \textit{отрицательные события} (англ. \textit{negative events}). Проблема заключается в том, что статус (то есть, сам факт существования, а также свойства) отрицательных событий до сих является предметом дискуссии \citep{asher2012reference,przepiorkowski1999negative,kamp2013discourse,de1999negation,fabergas2017building}. Ниже мы дадим краткое введение в проблематику отрицательных событий, а также покажем, что примеры из (\ref{azhNegativeEvents}) могут интерпретироваться соответсвующим образом.

\medskip

Мы начнем с того, какие случаи \citep{fabergas2017building} предлагают считать примерами отрицательных событий. \citep{fabergas2017building} утверждают, что отрицание может использоваться как для того, чтобы отрицать факт некоторого события, так и для утверждения факта отрицательного события:

\begin{exe}
    \ex The boy didn't eat. \begin{xlist}
        \ex \label{fabergasNegated} {\footnotesize \textit{Отрицание события}} \\ `It did not happen that the boy eat.'
        \ex \label{fabergasNegative} {\footnotesize \textit{Отрицательное событие}} \\ `It happened that the boy did not eat.'
    \end{xlist}
\end{exe}

Разница между интерпретациями в (\ref{fabergasNegated}) (\ref{fabergasNegative}) неочевидна, однако \citep{fabergas2017building} приводят ряд примеров из испанского языка, где отличия становятся более осязаемыми. Первый из таких примеров связан с инверсией шкал. В испанском языке существует конструкция \textit{<llegar a `стать/начать' + инфинитив>}

\medskip

\begingroup
    \fontsize{12pt}{12pt}\selectfont
    \begin{exe}
        \ex \gll Gritó e incluso dio una patada al hombre. \\
            he.yelled and even gave a kick to.the man \\
            \glt `He yelled and even kicked the man.'
    \end{exe}
\endgroup

\begin{exe}
    \ex \gll Gritó y llegó a dar una patada al hombre. \\
        he.yelled and became to to.give a kick to.the man \\
    \glt `He yelled and went so far as to kick the man.'
\end{exe}


\subsubsection{\textit{Аж} и передвижение в LF}

\begin{exe}
    \ex Я знал, что Лёша съездил аж $ [\textsc{в США}]_F $. \begin{xlist}
        \ex \textit{Узкое прочтение:} Поездка в США маловероятна (для Лёши).
        \ex \# \textit{Широкое прочтение:} Осведомленность о поездке Лёши в США маловероятна.
    \end{xlist}
\end{exe}

\subsection{{Хотя бы}} \label{xotyaBy}



\subsubsection{Семантика русского \textit{хотя бы}}

\begin{exe}
    \ex \label{onlyAssPres} Я съездил хотя бы $ [\textsc{в Турцию}]_F $ (а ты так всё лето в Москве и просидел). \begin{xlist}
        \ex \textit{Пресуппозиция:} Съездить в Турцию --- вероятная альтернатива.
        \ex \textit{Пресуппозиция:} Существуют более вероятные альтернативы.
        \ex \textit{Пресуппозиция:} Я не ездил ни одно из менее доступных мест, чем Турция.
    \end{xlist}
\end{exe}

\begin{exe}
    \ex \textsc{хотя бы(P)} \\ {\footnotesize Пусть $ P $ --- пропозиция, $ S $ --- упорядоченное множество контекстных альтернатив (шкала), $ \sigma $ --- контекстуальный стандарт, при  этом $ a >_{\sigma} b $ означает, что $ a $ находится выше $ b $ на шкале $ S $. Тогда:} \begin{xlist}
        \ex \label{xotyaByPresupLessThan} \textit{Пресуппозиция:} $ P \leq_{\sigma} \sigma $;
        \ex \label{xotyaByPresupNotMin} \textit{Пресуппозиция:} $ P >_{\sigma} S_{min} $;
        \ex \label{xotyaByPresupMax} \textit{Пресуппозиция:} Верно, что как максимум $ P $.
    \end{xlist}
\end{exe}

Для того, чтобы доказать корректность пресуппозиции (\ref{xotyaByPresupLessThan}), необходимо достаточно чётко определить понятие контекстуального стандарта. Выбранная формулировка пресуппозиции ($ P \leq_{\sigma} \sigma $) имеет своей целью объяснить следующее распределение приемлемости высказываний:

\begin{exe}
    \ex \begin{xlist}
        \ex \label{xotyaByLessThanGrade} \textit{В школе сочинения получают оценку от двойки до пятерки.} \\ \# Вася получил хотя бы $ [\textsc{пятерку}]_F $.
        \ex \label{xotyaByLessThanTrip} \textit{Обсуждаются варианты поездки на майские праздники в зависимости от их стоимости. Имеет место следующий порядок: \\
         (Остаться дома $ < $ Минск $ < $ $ \text{Прага}_{\sigma} $ $ \ll $ Вена $ \ll $ Нью-Йорк)}
         \begin{xlisti}
            \ex Дима поехал хотя бы в $ [\textsc{Прагу}]_F $.
            \ex \label{xotyaByLessThanTripBad} \# Дима поехал хотя бы в $ [\textsc{Вену}]_F $.
         \end{xlisti}
    \end{xlist}
\end{exe}

Пример (\ref{xotyaByLessThanGrade}) показывает, что употребление \textit{хотя бы} с наиболее высокой альтернативой недопустимо.

\medskip

В (\ref{xotyaByLessThanTrip}) поездка в Прагу является контекстуальным стандартом (на данный момент определим это как ``не является ни очень дешевым, ни очень дорогим вариантом''), в то время как отдых в Вене требует значительно больших трат. В таком случае (\ref{xotyaByLessThanTripBad}) приводит к недопустимости высказывания с \textit{хотя бы}.

\medskip

Очевидно, что контекстуальный стандарт может различаться от ситуации к ситуации; если бы (\ref{xotyaByLessThanTripBad}) было произнесено в разговоре между двумя состоятельными людьми (контекстуальный стандарт сдвигается на \textit{Вену}), высказывание стало бы приемлемым.

\medskip

Наличие пресуппозиции (\ref{xotyaByPresupNotMin}), требующей, чтобы пропозиция в фокусе не должна быть самой низкой на релевантной шкале, подтверждается следующими примерами:

\begin{exe}
    \ex \begin{xlist}
        \ex \textit{В школе сочинения получают оценку от двойки до пятерки.} \\ \# Вася получил хотя бы $ [\textsc{двойку}]_F $.
        \ex \textit{Обсуждаются подарки на дне рождения Тани.} \\ \# Андрей подарил хотя бы $ [\textsc{ничего}]_F $.
    \end{xlist}
\end{exe}

Наконец, пресуппозиция (\ref{xotyaByPresupMax}) исключает все более высокие альтернативы:

\begin{exe}
    \ex К нам на званый ужин пришел хотя бы $ [\textsc{премьер-министр}]_F $. (\# Президент тоже пришел.)
\end{exe}


\subsubsection{\textit{Хотя бы} и отрицание}

Примеры (\ref{xotyaByNeg}, \ref{xotyaByPPI}) показывают, что \textit{хотя бы} не может употребляться под сентенциальным отрицанием и в целом (как и \textit{аж}) ведет себя как PPI (кроме случая с двойным отрицанием в \ref{xotyaByPPIDoubleNeg}):

\begin{exe}
    \ex \label{xotyaByNeg} \textit{В школе сочинения получают оценку от двойки до пятерки.} \begin{xlist}
        \ex \label{xotyaByNegBad} \# Вася не получил хотя бы $ [\textsc{тройку}]_F $.
        \ex \label{xotyaByNegVP} Вася хотя бы не получил $ [\textsc{тройку}]_F $.
    \end{xlist}
\end{exe}

\begin{exe}
    \ex \label{xotyaByPPI} \begin{xlist}
        \ex \label{xotyaByPPIMatrixNeg} Я не говорил, что Вася получил хотя бы $ [\textsc{тройку}]_F $.
        \ex \label{xotyaByPPIAlwaysNeg} Вася не всегда получал хотя бы $ [\textsc{тройку}]_F $.
        \ex \# \label{xotyaByPPIDoubleNeg} Я не утверждаю, что Вася получил хотя бы $ [\textsc{тройку}]_F $.
    \end{xlist}
\end{exe}

Стоит заметить, что пример (\ref{xotyaByNegBad}) может оказаться допустимым в разговорной речи, если \textit{хотя бы} употребляется как энклитика; в этом случае высказывание интерпретируется так, как если бы \textit{хотя бы} модифицировало глагольную группу (\ref{xotyaByNegVP}).

\subsubsection{\textit{Хотя бы} и передвижение в LF}

\begin{exe}
    \ex Я знал, что Лёша съездил хотя бы $ [\textsc{в Мордовию}]_F $. \begin{xlist}
        \ex \textit{Узкое прочтение:} Поездка в Мордовию --- достаточно заурядное действие для Лёши.
        \ex \# \textit{Широкое прочтение:} Осведомленность о поездке Лёши в мордовию весьма .
    \end{xlist}
\end{exe}


\subsection{Хоть} \label{xot}

\subsubsection{Семантика \textit{хоть}}

Семантика русского \textit{хоть} представляет собой интересную вариацию семантики \textit{хотя бы}: в большинстве контекстов \textit{хоть} имеет ровно то же значение:

\begin{exe}
    \ex \label{onlyAssPres} Я съездил хоть $ [\textsc{в Турцию}]_F $ (а ты так всё лето в Москве и просидел). \begin{xlist}
        \ex \textit{Пресуппозиция:} Съездить в Турцию --- вероятная альтернатива.
        \ex \textit{Пресуппозиция:} Существуют более вероятные альтернативы.
        \ex \textit{Пресуппозиция:} Я не ездил ни одно из менее доступных мест, чем Турция.
    \end{xlist}
\end{exe}

\begin{exe}
    \ex \begin{xlist}
        \ex \textit{В школе сочинения получают оценку от двойки до пятерки.} \\ \# Вася получил хоть $ [\textsc{двойку}]_F $.
        \ex \textit{Обсуждаются подарки на дне рождения Тани.} \\ \# Андрей подарил хоть $ [\textsc{ничего}]_F $.
    \end{xlist}
\end{exe}

\begin{exe}
    \ex \begin{xlist}
        \ex \label{xotyaByLessThanGrade} \textit{В школе сочинения получают оценку от двойки до пятерки.} \\ \# Вася получил хоть $ [\textsc{пятерку}]_F $.
        \ex \label{xotyaByLessThanTrip} \textit{Обсуждаются варианты поездки на майские праздники в зависимости от их стоимости. Имеет место следующий порядок: \\
         (Остаться дома $ < $ Минск $ < $ $ \text{Прага}_{\sigma} $ $ \ll $ Вена $ \ll $ Нью-Йорк)}
         \begin{xlisti}
            \ex Дима поехал хотя бы в $ [\textsc{Прагу}]_F $.
            \ex \label{xotyaByLessThanTripBad} \# Дима поехал хоть в $ [\textsc{Вену}]_F $.
         \end{xlisti}
    \end{xlist}
\end{exe}

Это позволяет нам (предварительно) определить семантику \textit{хоть} следующим образом (дублируя семантику \textit{хотя бы}):

\begin{exe}
    \ex \textsc{хоть (P)} \\ {\footnotesize Пусть $ P $ --- пропозиция, $ S $ --- упорядоченное множество контекстных альтернатив (шкала), $ \sigma $ --- контекстуальный стандарт, при  этом $ a >_{\sigma} b $ означает, что $ a $ находится выше $ b $ на шкале $ S $. Тогда:} \begin{xlist}
        \ex \label{xotPresupLessThan} \textit{Пресуппозиция:} $ P \leq_{\sigma} \sigma $;
        \ex \label{xotPresupNotMin} \textit{Пресуппозиция:} $ P >_{\sigma} S_{min} $;
        \ex \label{xotPresupMax} \textit{Пресуппозиция:} Верно, что как максимум $ P $.
    \end{xlist}
\end{exe}

Приведенное выше определение утверждает, помимо всего прочего, что \textit{хоть} предпочитает достаточно низкие альтернативы. Однако в контекстах со снятой утвердительностью (англ. \textit{nonveridical}) для \textit{хоть} становится доступна высокая часть шкалы:

\begin{exe}
    \ex \begin{xlist}
        \ex Дима может нагрубить хоть $ [\textsc{премьер-министру}]_F $.
        \ex Дима может нагрубить хоть $ [\textsc{премьер-министру}]_F $?
        \ex Дима завтра нагрубит хоть $ [\textsc{премьер-министру}]_F $.
    \end{xlist}
\end{exe}

\subsubsection{\textit{Хоть} и передвижение в LF}

\begin{exe}
    \ex Я знал, что Лёша съездил хоть $ [\textsc{в Мордовию}]_F $. \begin{xlist}
        \ex \textit{Узкое прочтение:} Поездка в Мордовию --- достаточно заурядное действие для Лёши.
        \ex \# \textit{Широкое прочтение:} Осведомленность о поездке Лёши в мордовию весьма .
    \end{xlist}
\end{exe}

\newpage
\bibliography{bibliography}
\bibliographystyle{plainnat}

\end{document}

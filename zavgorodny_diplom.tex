\documentclass[a4paper, titlepage, 12pt]{article}
\usepackage[utf8]{inputenc}
\usepackage[russian]{babel}
\usepackage{mathtools}
\usepackage{fixltx2e}
\usepackage{stmaryrd}
\usepackage[perpage]{footmisc}
\usepackage{bm}
\usepackage{array}
\usepackage[table]{xcolor}
\usepackage{tikz}
\usepackage{float}
\usepackage{qtree}
\usepackage{pst-tree}
\usepackage[round, sort]{natbib}
\usepackage[nottoc,numbib]{tocbibind}
\usepackage{gb4e}

\noautomath
\restylefloat{table}
\renewcommand{\arraystretch}{1.5}
\newcommand*{\LargerCdot}{\raisebox{-0.25ex}{\scalebox{1.2}{$\cdot$}}}
\newcommand*{\LargeCdot}{\raisebox{-0.25ex}{\scalebox{1.8}{$\cdot$}}}
\newcolumntype{C}[1]{>{\centering\arraybackslash\hspace{0pt}}m{#1}}
\def\mathhyphen{{\hbox{-}}}

% COMPILE WITH: reset && pdflatex zavgorodny_diplom.tex && bibtex zavgorodny_diplom && pdflatex zavgorodny_diplom.tex && pdflatex zavgorodny_diplom.tex

\title{Семантика и дистрибуция фокусных частиц в русском языке}

\author{Андрей Завгородний}

\renewcommand*\contentsname{Summary}

\begin{document}

\begin{titlepage}

\newcommand{\HRule}{\rule{\linewidth}{0.5mm}} % Defines a new command for the horizontal lines, change thickness here

\center % Center everything on the page
 
%----------------------------------------------------------------------------------------
%   HEADING SECTIONS
%----------------------------------------------------------------------------------------

\textsc{\large Московский государственный университет имени \\ М.В. Ломоносова}\\[1.5cm] % Name of your university/college
\textsc{\large Филологический факультет}\\[0.5cm] % Major heading such as course name
\textsc{\large Отделение фундаментальной и прикладной лингвистики}\\[0.5cm] % Minor heading such as course title

~\\[2.0cm]

%----------------------------------------------------------------------------------------
%   TITLE SECTION
%----------------------------------------------------------------------------------------


{ \huge Семантика и дистрибуция скалярных фокусных частиц в русском языке}\\[0.4cm] % Title of your document
 
%----------------------------------------------------------------------------------------
%   AUTHOR SECTION
%----------------------------------------------------------------------------------------

~\\[3.0cm]

\begin{minipage}{0.4\textwidth}
\begin{flushleft} \large
Дипломная работа студента II курса магистратуры Андрея Олеговича Завгороднего \\
\end{flushleft}
\end{minipage}
~
\begin{minipage}{0.4\textwidth}
\begin{flushright} \large
Научный руководитель д.ф.н., проф. \\  Сергей Георгиевич Татевосов \\
\end{flushright}
\end{minipage}\\[5cm]

% If you don't want a supervisor, uncomment the two lines below and remove the section above
%\Large \emph{Author:}\\
%John \textsc{Smith}\\[2.5cm] % Your name

%----------------------------------------------------------------------------------------
%   DATE SECTION
%----------------------------------------------------------------------------------------

{\large Москва, 2017} % Date, change the \today to a set date if you want to be precise

%----------------------------------------------------------------------------------------
%   LOGO SECTION
%----------------------------------------------------------------------------------------

%\includegraphics{Logo}\\[1cm] % Include a department/university logo - this will require the graphicx package
 
%----------------------------------------------------------------------------------------

% \vfill % Fill the rest of the page with whitespace

\end{titlepage}

\thispagestyle{empty} 
\tableofcontents
\thispagestyle{empty}

\clearpage


\section[Введение]{Введение}

Введение.

\setcounter{page}{1}

\section{Фокус: семантика и прагматика}

В данной главе мы опишем наиболее значимые подходы к анализу семантики и прагматики фокуса. 

\subsection{Определение и общие сведения}

Фокус --- грамматическая категория, выделяющая в высказывании информационный компонент, являющийся новым или важным в том смысле, что говорящий не считает его разделенным между собой и слушающим \citep{Jackendoff1972}. Данное определение, разумеется, не может считаться исчерпывающим; ниже мы попытаемся привести необходимую и достаточную информацию о способах выражения и основных случаях употребления фокуса.

\medskip

Фокус может выражаться при помощи просодических (\ref{pitchAccentF}, фразовое ударение), синтаксических (\ref{cleftF}, клефт\footnote{Мнения относительно клефта как формы фокуса расходятся; см. \citep{ward2002grammar}}) или морфологических средств (\ref{morphemeF}), а также их комбинаций:

\begin{exe}
    \ex
    \begin{xlist}
        \ex \label{pitchAccentF} Я ищу \textbf{Машу}.
        \ex \label{cleftF} It is \textbf{John} we are looking for.
        \ex \label{morphemeF}
            {\footnotesize West Chadic}
            \gll Tí bà wúm-\textit{á} \textbf{kwálíngálá}. \\
                 \textsc{3sg} \textsc{prog} chew-\textsc{foc} colanut \\
            \glt `He is chewing \textbf{colanut}.' \citep[ex.\ 3b]{hartmann}
    \end{xlist}
\end{exe}

\medskip

Фокус имеет прямое отношение к семантике высказывания, поскольку способен влиять на его истинностное значение. Ниже приведен классический пример смыслоразличительной функции фокуса из английского языка:

\begin{exe}
    \ex \begin{xlist} \label{truthValues}
        \ex \label{truthValues1} John only introduced [\textsc{Bill}]\textsubscript{F} to Sue.
        \ex \label{truthValues2} John only introduced Bill to [\textsc{Sue}]\textsubscript{F}.
    \end{xlist}
\end{exe}

В примере (\ref{truthValues1}) утверждается, что единственным человеком, которого представили Сью, был Билл. В (\ref{truthValues2}), напротив, говорится, что Билла представили только Сью (в то время как Сью могли также представить Джона, Гарри, etc.)

\medskip

Фокус также может влиять на уместность высказывания в дискурсе:

\begin{exe}
    \ex \begin{xlist} \label{discourseSensitivity}
        \ex Эту книгу продают на Озоне.
        \ex Ну и что? Эту книгу продают даже в [\textsc{Читай-городе}]\textsubscript{F}.
        \ex [\#]{Ну и что? Эту книгу даже [\textsc{продают}]\textsubscript{F} в Читай-городе.}
    \end{xlist}
\end{exe}

\medskip

Здесь и далее мы будем выделять часть высказывания, несущую на себе фразовое ударение, \textsc{малыми заглавными} буквами; составляющая, которая несет на себе фокус, будет обозначаться нижним индексом F. Указанное разграничение неслучайно: фразовое ударение в одном и том же месте высказывания может интерпретироваться как фокус на составляющих разного размера:

\begin{exe}
    \ex \begin{xlist} \label{pitchAccentVSfocus}
        \ex Марк купил книгу о [\textsc{неомарксизме}]\textsubscript{F}. \\ \textit{(О чем была книга, которую купил Марк?)}
        \ex Марк купил книгу [о \textsc{неомарксизме}]\textsubscript{F}. \\ \textit{(Какую книгу купил Марк?)}
        \ex Марк купил [книгу о \textsc{неомарксизме}]\textsubscript{F}. \\ \textit{(Что купил Марк?)}
        \ex Марк [купил книгу о \textsc{неомарксизме}]\textsubscript{F}. \\ \textit{(Что сделал Марк?)}
        \ex ~[Марк купил книгу о \textsc{неомарксизме}]\textsubscript{F}. \\ \textit{(Что произошло?)}
    \end{xlist}
\end{exe}

\medskip

Проблеме различных стратегий проекции фокуса посвящена обширная литература. Существуют две основных теории реализации фокуса на составляющей: проекции \textit{сверху} и проекции \textit{снизу} \citep{winkler1997focus}. В первом случае \citep{gussenhoven1983focus} утверждается, что признак фокуса присваивается наиболее высокому узлу и затем просачивается ``вниз'', пока не находит наиболее подходящее место для реализаци; во втором случае \citep{selkirk1984phonology, selkirk1996prosodic} признак фокуса изначально находится на составляющей с фразовым ударением, после чего проецируется вверх, маркируя сферу действия фокуса.

\subsection{Семантика фокуса}

В данном разделе мы рассмотрим существующие подходы к анализу семантики фокуса. В \ref{alternativeSemantics} мы изложим основные положения \textit{семантики альтернатив} \citep{rooth1985association,rooth1992theory}; в \ref{structuredMeanings} мы рассмотрим иной подход, названный теорией \textit{структурированных значений} \citep{Cresswell1985}, а также проведем его сравнение с семантикой альтернатив.

\subsubsection{Семантика альтернатив \citep{rooth1985association,rooth1992theory}} \label{alternativeSemantics}

Семантика альтернатив \citep{rooth1985association,rooth1992theory} является одним из наиболее известных и распространенных подходов к анализу семантики фокуса. Утверждается, что фокус привносит в процесс интерпретации высказывания \textit{множество альтернатив}, существующих параллельно с ``обычной'' семантикой высказывания.

\medskip

Этот подход близок к анализу семантики вопросов в работах \citep{hamblin1973questions,karttunen1977syntax}, где интерпретация вопроса \textit{What does Andrew smoke?} будет выглядеть как множество ответов $ \{ smoke(A, x) \ | \ object(x) \} $. В литературе разделились мнения относительно того, какого рода ответы должны входить в такое множество: должны ли это быть истинные и возможные \citep{hamblin1973questions}, истинные \citep{karttunen1977syntax} или просто возможные \citep{groenendijk1985semantics} ответы. В любом случае, множество $ \{ smoke(A, x) \ | \ object(x) \} $ --- то, что в \citep{rooth1985association} называется \textsc{фокусным значением} (англ. \textit{focus semantic value}) выражения $ Andrew\ smokes\ [\textsc{cigarettes}]_{F} $.

\medskip

Понятие фокусного значения используется для интерпретации чувствительных к фокусу элементов вроде английского \textit{only} или русского \textit{только}. Предположим, чтот $ S $ --- предложение, содержащее \textit{only} (например, $ Andrew\ only\ smokes\ [\textsc{cigarettes}]_{F} $), а $ M $ и $ R $ --- его фокусное и обычное значения соответственно. $ only(S) $ означает, что все альтернативы, кроме $ S $, не верны. Мы можем записать сказанное формально:

\begin{exe}
    \ex \label{simpleAlternatives} $ \forall P \in M, true(p) \rightarrow P = R $
\end{exe}

В таком случае для предложения выше мы будем иметь следующую интерпретацию:

\begin{exe}
    \ex \begin{xlist}
        \ex \textit{Every true proposition stating that Andrew smokes some object is the proposition that Andrew smokes cigarettes.}
        \ex $ \forall P \in \{ smoke(A, x) \ | \ object(x) \}, true(p) \rightarrow P = smoke(A, c) $
    \end{xlist}
\end{exe}

Однако приведенная в (\ref{simpleAlternatives}) интерпретация фокуса не является удовлетворительной в том смысле, что не учитывает \textit{контекст} высказывания, который может значительно ограничивать исходное множество всех возможных альтернатив. Например, очевидно, что в примере (\ref{restrictedAlternatives}) выбор происходит между двумя определенными альтернативами:

\begin{exe}
    \ex \begin{xlist} \label{restrictedAlternatives}
        \ex A: What does Andrew smoke: cigarettes or pipe?
        \ex B: He smokes only [\textsc{cigarettes}]\textsubscript{F}.
    \end{xlist}
\end{exe}

Для решения этой проблемы \citep{rooth1992theory} предлагает использовать понятие множества \textit{контекстно уместных} альтернатив, которое должно являться подмножеством фокусного значения. В таком случае значение $ only(S) $ можно записать следующим образом:

\begin{exe}
    \ex \label{salientAlternatives}
        $ \llbracket only(S) \rrbracket = \forall P \in ALT\ [true(P) \rightarrow (P = S')] $, где S' --- это обычное значение высказывания $ S $, а $ ALT $ --- множество контекстно уместных альтернатив, причем $ ALT $ является подмножеством фокусного значения $ S $.
\end{exe}

Для примера (\ref{restrictedAlternatives}) $ ALT $ будет множеством, состоящим из двух пропозиций: $ \{ Andrew\ smokes\ cigarettes,\ Andrew\ smokes\ pipe \} $.

\medskip

Проблемой семантики альтернатив в её форме, изложенной в \citep{rooth1985association}, является то, что с её помощью невозможно описать механизм ассоциации с фокусом в целом; нам необходимо прописывать правила ассоциации с фокусом для каждого типа чувствительной конструкции по отдельности. \citep{rooth1992theory} предлагает называть такие теории интерпретации фокуса \textit{слабыми}. Например, в интерпретации для $ only(S) $ используется отсылка к $ ALT $, значение которого определяется \textit{фокусным} значением $ S $.

\medskip

\citep{rooth1992theory} разрабатывает собственную \textit{сильную} теорию интерпретации фокуса, которая не требует формулировки правил для отдельных конструкций. Перечислим типы конструкций, чувствительных к фокусу:

\begin{exe}
    \ex \begin{xlist} \label{focusConstructions}
        \ex \textbf{Фокусные наречия.} Если $ C $ --- область квантификации фокусного наречия с аргументом $ \alpha $ , то $ C \subseteq \llbracket \alpha \rrbracket^{F}. $ \\ Пример: \textit{only}.
        \ex \textbf{Контрастирующие выражения.} Если выражение $ \alpha $ контрастирует с выражением $ \beta $, то $ \llbracket \beta \rrbracket^{o} \in \llbracket \alpha \rrbracket^{F}. $ \\ Пример: [\textsc{Русский}]\textsubscript{F} \textit{солдат встречает} [\textsc{американского}]\textsubscript{F} \textit{солдата.}
        \ex \textbf{Выражения со скалярной импликацией.} Если $ C $ --- упорядоченное множество, использующееся для формирования импликатуры выражения $ \alpha $, то $ C \subseteq \llbracket \alpha \rrbracket^{F}. $ \\ Пример: \textit{Как прошел экзамен? --- Ну, я} [\textsc{сдал}]\textsubscript{F} (\textit{сдать на отлично}$ \implies $\textit{сдать}).
        \ex \textbf{Выражения типа ``вопрос--ответ''.} Для пары вопрос--ответ $ \langle \beta, \alpha \rangle $ верно, что $ \llbracket \beta \rrbracket^{o}  \subseteq \llbracket \alpha \rrbracket^{F}. $ \\ Пример: см. пример (\ref{pitchAccentVSfocus}).
    \end{xlist}
\end{exe}

Здесь $ \llbracket . \rrbracket^{o} $ означает ``обычное'' семантическое значение выражения.

\medskip

Далее замечается, что в каждой из приведенных конструкций речь идет о том, что некий семантический объект является либо элементом, либо подмножеством фокусного значения высказывания. Разница между этими случаями заключается в типе (в широком смысле) семантического объекта, причем перечисление возможных типов не должно входить в собственно теорию \textit{фокуса}.

\medskip

На основании этих рассуждений репертуар семантики альтернатив сокращается до единственного оператора $ \sim $ \citep{rooth1992theory}:

\begin{exe} 
    \ex \label{tildaOperator} Если $ \phi $ --- синтаксическое выражение, то $ \phi \sim C $ вводит пресуппозицию, что:
    \begin{xlist}
        \ex $ C \subseteq \llbracket \phi \rrbracket^{F} $, содержащим $ \llbracket \phi \rrbracket^{o} $ и еще как минимум один элемент, если $ C $ --- множество;
        \ex $ C \in \llbracket \phi \rrbracket^{F} $ и $ C \neq \phi $, если $ C $ --- индивидуальный элемент.
    \end{xlist}
\end{exe}

Преимуществом такого правила для интерпретации фокуса является тот факт, что мы не должны упоминать в нём конкретные конструкции, вступающие во взаимодействие с фокусом. Фокус просто вводит в интерпретацию \textit{некую} свободную переменную $ C $ и устанавливает ограничения на семантику (семантический тип) только \textit{этой переменной}. Более того, при задании семантического значения ``чувствительных к фокусу'' элементов мы можем в принципе обойтись без упоминания понятия фокуса; например, в \citep{rooth1992theory} для $ only(V\!P) $ предлагается следующая интерпретация:

\begin{exe}
    \ex \label{onlyVPredefined} $ \llbracket only\ V\!P \rrbracket = \lambda x.[\forall P [P \in C \land P(x) \rightarrow P = V\!P ]] $
\end{exe}

Здесь важно еще раз подчеркнуть, что $ C $ --- просто некая свободная переменная. Удаляя упоминания фокуса из семантического значения, мы также решаем проблему, связанную со случаями употребления $ only $ без эксплицитно выраженного фокуса (подробнее об этом см. раздел \ref{focusApproaches}).

\medskip

Приведем пример интерпретации следующего выражения с использованием определения из (\ref{tildaOperator}):

\begin{exe}
    \ex $ Mary\ only\ introduced\ [\textsc{Bill}]_{F}\ to\ Sue. $
\end{exe}

Дерево синтаксического разбора для этого примера будет выглядеть приблизительно так:

\begin{exe}
  \ex \label{tildaTree} \Tree [.S [.NP Mary ] [.VP  { $ only(C) $ } [.VP \qroof{$ introduced\ [Bill]_{F}\ to\ Sue $}.VP {$ \sim C $} ] ] ]
\end{exe}

Как видно из (\ref{tildaTree}), фокус интерпретируется на уровне VP. $ C $ здесь является свободной переменной и ведет себя как ``местоимение'', отсылающее ко множеству доступных альтернатив; после окончания деривации оно должно быть заполнено альтернативами вида $ \{ Introduce(x, \bm{Bill}, Sue),\\Introduce(x, \bm{Jack}, Sue), Introduce(x, \bm{Kyle}, Sue), ... \} $ (о фиксации этого множества см. ниже в (\ref{fixingC})).

\medskip

Под семантическим значением VP понимается обычное лямбда-выражение вида $ \lambda z.\lambda y.\lambda x.[Introduce(x, y, z)] $. Если подставить это выражение в (\ref{onlyVPredefined}), имеем следующую деривацию:

\begin{exe}
    \ex \label{} $ \lambda x_{1} .[\forall P [P \in C \land P(x_{1}) \rightarrow P = \lambda x_{2}.Introduce(x_{2}, Bill, Sue) ] $ \\ $ [\forall P [P \in C \land P(Mary) \rightarrow P = \lambda x_{2}.Introduce(x_{2}, Bill, Sue) ] $
\end{exe}

Как мы видим, результатом интерпретации на уровне семантики является выражение со свободной переменной $ C $. В \citep{rooth1992theory} предлагается считать, что окончательная фиксация значения этой переменной происходит на уровне прагматики. В примере (\ref{fixingCtext}) множество альтернатив дополнительно ограничено контекстом:

\begin{exe}
    \ex \label{fixingCtext} \begin{xlist}
        \ex $ Who\ did\ Mary\ introduce\ to\ Sue,\ Bill\ or\ John? $
        \ex $ Mary\ only\ introduced\ [\textsc{Bill}]_{F}\ to\ Sue. $
    \end{xlist}
\end{exe}

Результат интерпретации (\ref{fixingCtext}) на уровне прагматики будет выглядеть так: 

\begin{exe}
    \ex \label{fixingC} $ \exists C [ C = \{ Introduce(x, \bm{Bill}, Sue), Introduce(x, \bm{Jack}, Sue) \} \\ \land [\forall P [P \in C \land P(Mary) \rightarrow P = \lambda x_{2}.Introduce(x_{2}, Bill, Sue) ]] $
\end{exe}

\subsubsection{Структурированные значения \citep{klein1982intonation,krifka1992compositional}} \label{structuredMeanings}

Теория \textit{структурированных значений} \citep{klein1982intonation,krifka1992compositional} похожа на семантику альтернатив тем, что предполагает наличие двух компонент в интерпретации фокусного высказывания. Однако если в семантике альтернатив этими компонентами были стандартное и фокусное значение, то структурированным значением является кортежем из \textit{вопроса} (фона, англ. \textit{background}) и простого \textit{ответа} (фокуса):

\begin{exe}
    \ex \label{structuredIntro} \begin{xlist}
        \ex $ Kyle\ hates\ the\ [\textsc{fatboy}]_{F} $
        \ex $ \langle \lambda x.Hate(Kyle, x), \imath\ f\!atboy \rangle $
    \end{xlist}
\end{exe}

Обычное значение выражения получается путем функционального применения фона к фокусу. Если фокус находится на предикате $ Kyle\\~ [\textsc{hates}]_{F}\ the\ f\!atboy $, то значение должно выглядеть как

\begin{exe}
    \ex \label{structuredIntroPredicate} $ \langle \lambda P.[P(Kyle, \imath\ f\!atboy)], hate \rangle $
\end{exe}

В примерах (\ref{structuredIntro}, \ref{structuredIntroPredicate}) можно наблюдать не совсем типичную репрезентацию значения высказывания. Очевидно, что при классическом композициональном подходе невозможно добиться лямбда-абстракции внутреннего аргумента глагола при фиксированном внешнем аргументе (i.e., $ \lambda x.Hate(Kyle, x) $).

\medskip

В \citep{krifka1992compositional} для решения данной проблемы предлагается достаточно сложная альтернатива стандартному правилу функционального применения. Мы приведем (в целях ознакомления) определение семантических типов и формулировку \textit{расширенного} функционального применения, однако в дальнейшем будем использовать упрощенную нотацию, предлагаемую в \citep{beaver2008sense}.

\medskip


\begin{exe}
    \ex Семантические типы:
    \begin{xlist}
        \ex $ e $ есть тип сущностей, $ t $ --- тип истинностных значений.
        \ex Если $ \sigma, \tau $ --- типы, то:
        \begin{xlisti}
            \ex $ \sigma(\tau) $ есть тип функций от $ \sigma- $денотаций к $ \tau- $денотациям;
            \ex $ \sigma \LargeCdot \tau $ есть тип списка $ \sigma- $денотаций и $ \tau- $денотаций;
            \ex $ \langle \sigma, \tau \rangle $ есть тип структуры фон---фокус.
            
        \end{xlisti}
    \end{xlist}
\end{exe}


\begin{exe}
    \ex Расширенное функциональное применение ``(\ )'':
    \begin{xlist}
        \ex Если $ \alpha $ имеет тип $ \sigma(\tau) $ и $ \beta $ имеет тип $ \sigma $, то $ \alpha(\beta) $ имеет тип $ \tau $ и означает функциональное применение;
        \ex \textit{{\footnotesize Наследование фокуса от оператора:}}\\
        Если $ \langle \alpha, \beta \rangle $ имеет тип $ \langle (\sigma)(\tau)\mu, \sigma^{\LargerCdot} \rangle $ и $ \gamma $ имеет тип $ \tau $, то $ \langle \alpha, \beta \rangle(\gamma) $ имеет тип $ \langle (\sigma)\mu, \sigma^{\LargerCdot} \rangle $ и интерпретируется как $ \langle \lambda X_{\sigma}.[\alpha(X)(\gamma)], \beta \rangle $;
        \ex \textit{{\footnotesize Наследование фокуса от аргумента:}}\\
        Если $ \gamma $ имеет тип $ (\sigma)\tau $ и $ \langle \alpha, \beta \rangle $ имеет тип $ \langle (\mu)\sigma, \mu^{\LargerCdot} \rangle $, то $ \gamma(\langle \alpha, \beta \rangle) $ имеет тип $ \langle (\mu)\tau, \mu^{\LargerCdot} \rangle $;
        \ex \textit{{\footnotesize Наследование фокуса от оператора и аргумента:}}\\
        Если $ \langle \alpha, \beta \rangle $ имеет тип $ \langle (\sigma)(\tau)\mu, \sigma^{\LargerCdot} \rangle $ и $ \langle \gamma, \delta \rangle $ имеет тип $ \langle (\upsilon)\tau, \upsilon^{\LargerCdot} \rangle $, то $ \langle \alpha, \beta \rangle(\langle \gamma, \delta \rangle) $ имеет тип $ \langle (\sigma \LargeCdot \upsilon)\mu, \sigma^{\LargerCdot} \LargeCdot \upsilon^{\LargerCdot} \rangle $ и интерпретируется как $ \langle \lambda X_{\sigma} \LargeCdot Y_{\upsilon}.[\alpha(X)(\gamma(Y))], \beta \LargeCdot \delta \rangle $.
    \end{xlist}
\end{exe}

\medskip

Как было замечено выше, \citep{beaver2008sense} предлагают упрощенное представление деривации в рамках теории структурированных значений; мы будем придерживаться этой упрощенной нотации.

\medskip

Для случаев, когда выражение содержит несколько элементов, несущих фокус, требуется введение операций над \textit{составными объектами} \citep{krifka1992compositional}. Так, для интерпретации выражения $ Mary\ introduced\ [\textsc{Bill}]_{F}\\to\ [\textsc{Sue}]_{F} $ мы должны ввести операцию лямбда-абстракции относительно списков переменных:

\begin{exe}
    \ex $ \langle \lambda [x,y].Introduce(Mary, x, y), [Bill, Sue] \rangle $
\end{exe}

Также необходимо ввести правила для функционального применения выражения $ A $ к $ B $, если оба этих выражения являются кортежами \textit{фон, фокус}. Для этого введем операцию конкатенации списков:

\begin{exe}
    \ex $ [x_1, ..., x_i] \sqcap [y_1, ..., y_j] = [x_1, ..., x_i, y_1, ..., y_j] $
\end{exe}


Тогда для выражений $ A $ и $ B $, имеющих структурированные значения $ \langle \alpha_b, \alpha_f \rangle $ и $ \langle \beta_b, \beta_f \rangle $ соответственно, применение $ AB $ будет иметь следующий вид:

\begin{exe}
    \ex $ \langle \lambda[x, y].\alpha_b(x)(\beta_b(y)), \alpha_f \sqcap \beta_f \rangle $
\end{exe}

Для удобства нотации введем следующие обозначения. Пусть вырожденные случаи лямбда-абстракции и функционального применения не будут иметь никакого эффекта: $ \lambda \langle\ \rangle\!=\!\phi $ и $ \phi(\langle\ \rangle)\!=\!\phi $. Тогда будем считать, что ``$ \langle \phi $'' означает ``$ \langle \phi, [\ ] \rangle $'', т.е. структурированное значение с фоном, но без фокуса; также положим, что ``$ \phi \rangle $'' означает `` $ \langle \lambda X.X, \phi \rangle $'', т.е. фокус с тривиальным фоном (чисто фокусное значение). 

\medskip

Рассмотрим теперь деривацию\footnote{Еще раз отметим, что приведенная деривация является упрощенной записью достаточно сложного формализма, разработанного в \citep{krifka1992compositional}.} выражения $ Andrew\ gave\ [\textsc{Kyle}]_{F}\ cigarettes $ \citep[4.15]{beaver2008sense}:

\begin{exe}
    \ex \label{derivingStruct}
        \begin{flalign*}
            Kyle                                    &\mapsto \langle k &                                    \\
            [Kyle]_{F}                              &\mapsto k \rangle                                      \\
            gave                                    &\mapsto \langle Give                                   \\
            gave\ [Kyle]_{F}                        &\mapsto \langle Give, k \rangle                        \\
            cigarettes                              &\mapsto \langle c                                      \\
            gave\ [Kyle]_{F}\ cigarettes            &\mapsto \langle \lambda \beta. Give(\beta)(c), k \rangle       \\
            Andrew                                  &\mapsto \langle a                                      \\
            Andrew\ gave\ [Kyle]_{F}\ cigarettes    &\mapsto \langle \lambda x. Give(a)(x)(c), k \rangle    \\
        \end{flalign*}
\end{exe} 

Далее, для конфигурации $ \llbracket only(V\!P) \rrbracket $ предлагается следующая интерпретация:

\begin{exe}
    \ex $ \lambda[B,F].\langle \lambda x.\forall \gamma [B(\gamma)(x) \rightarrow \gamma = F ] $
\end{exe}

В таком случае мы имеем следующую интерпретацию выражения $ Andrew\ only\ gave\ [\textsc{Kyle}]_{F}\ cigarettes $:

\begin{exe}
    \ex \label{derivingStruct}
        \begin{align*}
            only                                          &\mapsto \lambda[B,F].\langle \lambda x.\forall \gamma [B(\gamma)(x) \rightarrow \gamma = F ] \\
            only\ gave\ [Kyle]_{F}\ cigarettes            &\mapsto \langle \lambda x.\forall \gamma [\lambda [\beta.Give(\beta)(c)](\gamma)(x) \rightarrow \gamma = b] \\
                                                          &\mapsto \langle \lambda x.\forall \gamma [Give(\gamma)(c)(x) \rightarrow \gamma = b] \\
            Andrew                                        &\mapsto \langle a                                      \\
            Andrew\ only\ gave \\ [Kyle]_{F}\ cigarettes  &\mapsto \langle \forall \gamma [Give(\gamma)(c)(a) \rightarrow \gamma = b ]   \\
        \end{align*}
\end{exe} 


\subsection{Прагматика фокуса}\label{focusPragmatics}

В данном разделе мы кратко рассмотрим подходы к анализу фокуса с точки зрения прагматики.

\subsubsection{О классификации подходов к анализу фокуса} \label{focusApproaches}

В разделе \ref{alternativeSemantics} было упомянуто, что \citep{rooth1992theory} различал \textit{слабые} и \textit{сильные} теории фокуса. Слабые теории \citep{rooth1985association} отличаются тем, что в них чувствительные к фокусу элементы напрямую обращаются к фокусному значению отдельных составляющих (\ref{weak}), в то время как сильные \citep{rooth1992theory} избегают этого (\ref{strong}). \citep{beaver2008sense} используют термины \textit{конвенциональная} и \textit{свободная} ассоциация с фокусом, в целом соответствующие слабой и сильной ассоциации в \citep{rooth1992theory}.

\begin{exe}
    \ex \begin{xlist}
        \ex \label{weak} $ \llbracket only\ V\!P \rrbracket = \lambda x.[\forall P [P \in \llbracket V\!P \rrbracket^{F} \land P(x) \rightarrow P = V\!P]] $
        \ex \label{strong} $ \llbracket only\ V\!P \rrbracket = \lambda x.[\forall P [P \in C \land P(x) \rightarrow P = V\!P ]] $
    \end{xlist}
\end{exe}

Очевидно, что в (\ref{strong}) свободная переменная $ C $ чаще всего определяется набором фокусных альтернатив, вводящихся в пресуппозицию на этапе синтаксической деривации (см. (\ref{tildaTree})). Преимуществом данного подхода (помимо эстетических соображений) является то, что в сильной теории мы не вынуждены утверждать, что чувствительные к фокусу выражения могут употребляться только в контекстах с эксплицитным грамматическим фокусом; существование \textit{безакцентного} фокуса может служить аргументом в пользу сильных теорий:

\begin{exe}
    \ex A: Мне кажется, все знали, что Кламм общается только\footnote{Мы полагаем, что в релевантных для данного обсуждения аспектах русское \textit{только} ведет себя так же, как английское \textit{only}.} с [\textsc{Фридой}]\textsubscript{F}. \\
        B: Даже [\textsc{К.}]\textsubscript{F} знал, что Кламм общается только с [\textsc{Фридой}]\textsubscript{SOF}, но это его мало заботило.
\end{exe}

Второе упоминание Фриды в ответе \textit{В} не несет на себе эксплицитного фразового ударения, но употребление \textit{только} в этом контексте все равно является грамматичным; это явление называется вторичным фокусом (англ. \textit{second occurence focus}) \citep{partee199911}. В (\ref{preAccentlessOnly}) дан еще однин пример безакцентного употребления \textit{only} (здесь вторичный фокус предшествует фразовому ударению):

\begin{exe} 
    \ex \label{preAccentlessOnly} A: I hear that John only gave [\textsc{a book}]\textsubscript{F} to Mary. \\
    B: True, but John only gave [\textsc{a book}]\textsubscript{SOF} SOF to [\textsc{many people}]\textsubscript{F}.
\end{exe}

Стоит заметить также, что в \citep{beaver2008sense} нередко говорится о чисто прагматическом (англ. \textit{purely pragmatic}) подходе к анализу фокуса (так, например, говорится о подходе \citep{roberts1996information}, см. раздел \ref{robertsFocus}); можно считать, что здесь имеется в виду то же, что и в случае свободной (сильной) ассоциации. В этом смысле теория \citep{rooth1992theory} является не менее ``прагматической'', чем \citep{roberts1996information}, однако в последней гораздо больше внимания уделяется непосредственно механизму фиксации множества альтернатив в прагматике. По этой причине мы приводим изложение \citep{roberts1996information} именно в разделе о прагматике фокуса.

\subsubsection{Прагматический механизм ассоциации с фокусом \citep{roberts1996information,beaver2008sense}} \label{robertsFocus}

Подход \citep{roberts1996information} является чисто прагматической (сильной) теорией фокуса: все чувствительные к фокусу выражения ассоциируются свободно. Так, \textit{only} получает интерпретацию, в которой рестриктором для квантификации является свободная переменная (ср. с интерпретацией в \citep{rooth1992theory}):

\begin{exe}
    \ex $ Andrew\ only\ smokes\ [\textsc{cigarettes}]_{F}\ on\ Fridays. $
    \begin{xlist}
            \ex \label{robertsOnly} $ \forall P [P \in R \land true(P) \rightarrow P = smoke(Andrew, cigarettes)] $
            \ex $ R = \{smoke(Andrew, x)\ |\ x \in D \} $
    \end{xlist}
\end{exe}

В \citep{roberts1996information} подробно обсуждается механизм фиксации значения переменной $ R $ (множества контекстно доступных альтернатив). Предлагается формальное развитие идеи \citep{rooth1985association} о том, что это множество должно зависеть от \textit{обсуждаемого вопроса} (англ. \textit{question under discussion}).

\medskip

Прежде всего \citep{roberts1996information} дает собственное определение семантики вопроса. Вопрос интерпретируется как абстракция по каждому из содержащихся в нем \textit{wh}-слов: $ \llbracket W\!ho\ arived? \rrbracket = who(\lambda x.[x\ arrived]) $. Множество Q-альтернатив вопроса задается как применение каждого из абстрагированных \textit{wh}-слов ко всем доступным сущностям в модели.

\begin{exe}
    \ex \label{robertsQuestion} Множество \textbf{Q-альтернатив} вопроса:\\ $ Q \mathhyphen alt(\alpha) = \{ P\ |\ \exists u^{i-1}_{\in D}, ..., u^{i-n}_{\in D} [ P = \llbracket \beta \rrbracket (u^{i-1}) ... (u^{i-n}) ]]\} $,
    где $ \alpha $ имеет вид $ wh_{i-1}, ..., wh_{i-n}(\beta) $; $ \{ wh_{i-1}, ..., wh_{i-n} \} $ --- (возможно, пустое) множество \textit{wh}-слов в $ \alpha $, а $ D $ --- область определения модели (множество всех людей для \textit{who} и множество всех вещей для \textit{what}).
\end{exe}

Далее выделяются критерии, которым должно соответствовать высказывание с фокусом; этими критериями являются \textbf{релевантность} и \textbf{конгруэнтность}:

\begin{exe}
    \ex \begin{xlist}
            \ex Высказывание $ m $ \textbf{релевантно} для обсуждаемого вопроса  $ q $ в одном из двух случаев:
                \begin{xlisti}
                    \ex $ m $ является частичным ответом на вопрос $ q $ ($ m $ --- утверждение);
                    \ex $ m $ является частью стратегии ответа на вопрос $ q $ ($ m $ --- вопрос).
                \end{xlisti}
            \ex Ответ $ \beta $ \textbf{конгруэнтен} вопросу $\!? \alpha $, если его фокусное значение совпадает со множеством Q-альтернатив вопроса: $ \llbracket \beta \rrbracket^{F} = Q \mathhyphen alt(\alpha) $.
    \end{xlist}
\end{exe}

В таком случае значением $ R $ для (\ref{fuckedUpExpression}) будет являться множество Q-альтернатив для вопроса в (\ref{fuckedUpQuestion}):

\begin{exe}
    \ex \begin{xlist}
            \ex \label{fuckedUpQuestion} Какой предмет/предметы является таковым, что Андрей не имеет других свойств, кроме курения этого предмета/предметов по пятницам?\footnote{Для применения определения вопроса из (\ref{robertsQuestion}) предполагается соответствие английского \textit{which} русскому \textit{какой}.}
            \ex \label{fuckedUpExpression} По пятницам Андрей курит только [\textsc{сигареты}]\textsubscript{F}.
    \end{xlist}
\end{exe}

\citep{roberts1996information} замечает, что вопрос в (\ref{fuckedUpQuestion}), хотя и полностью соответствует определениям (\ref{robertsOnly}, \ref{robertsQuestion}), не может быть задан в исходной форме: при условии, что Андрей существует, помимо курения сигарет по пятницам у него должно быть хотя бы свойство самоидентичности.

\medskip

Утверждается, что вследствие максимы качества по Грайсу (англ. \textit{Gricean Maxim of Quality}) вопрос (\ref{fuckedUpQuestion}) может быть употреблен в дискурсе только в том случае, если подразумевается его релевантность для вопроса (\ref{robertsGeneralQuestion}):

\begin{exe} 
    \ex \label{robertsGeneralQuestion} Какой предмет/предметы является таковым, что Андрей имеет свойство курения этого предмета/предметов по пятницам? \\
    (Что курит Андрей по пятницам?)
\end{exe}

Иными словами, область определения (англ. \textit{domain}) \textit{only} сокращается до множества предикатов курения кем-то чего-то по пятницам. В таком случае (\ref{fuckedUpQuestion}) приобретает следующую трактовку:

\begin{exe}
    \ex \label{fuckedUpQuestionCorrected} Какой предмет/предметы является таковым, что Андрей не имеет других свойств курения предмета/предметов по пятницам, кроме курения этого предмета/предметов по пятницам?
\end{exe}

\citep{roberts1996information} утверждает, что поскольку любой полный ответ на (\ref{fuckedUpQuestionCorrected}) является также ответом на (\ref{robertsGeneralQuestion}), возможен дискурс в (\ref{robertsGeneralizedDiscourse}):

\begin{exe}
    \ex \label{robertsGeneralizedDiscourse} \begin{xlist}
            \ex \label{robertsGeneralizedDiscourseQuestion} Что курит Андрей по пятницам?
            \ex \label{robertsGeneralizedDiscourseAnswer} По пятницам Андрей курит только [\textsc{сигареты}]\textsubscript{F}.
    \end{xlist}
\end{exe}

\citep{kadmon2001formal} замечает, что подобное рассуждение не дает нам инструментов для точного задания множества альтернатив. Предположим, что в примере (\ref{robertsOnly}) $ R $ содержит не только предсказываемый \citep{roberts1996information} набор альтернатив, но также альтеративу, звучащую как \textit{Андрей курит сигареты по понедельникам}. В таком случае вопрос в (\ref{fuckedUpQuestionCorrected}) будет иметь следующий вид:

\begin{exe}
    \ex \label{robertsFuckedUpQuestion} Какой предмет/предметы является таковым, что либо Андрей не имеет других свойств курения предмета/предметов по пятницам, кроме курения этого предмета/предметов по пятницам, либо Андрей ничего не курит по пятницам и курит сигареты по понедельникам?
\end{exe}

Несмотря на то, что такой вопрос звучит очень странно, любой полный ответ на (\ref{robertsFuckedUpQuestion}) будет также являться ответом на (\ref{robertsGeneralizedDiscourseQuestion}), а значит для \citep{roberts1996information} значение $ R $, определенное таким образом, будет являться удовлетворительным.

\subsubsection{Анализ фокуса в фреймворке DRT \citep{geurts2004interpreting}}

\section{Фокус и фокусные частицы}

В данном разделе мы дадим общее описание семантики и прагматики фокусных частиц их классификацию (\ref{focusParticles}); мы также подробно рассмотрим класс \textit{скалярных} фокусных частиц (\ref{scalarOperators}).

\subsection{Фокусные частицы: обзор} \label{focusParticles}

Обзор.

\subsection{Скалярные фокусные операторы} \label{scalarOperators}

\citep{gast2011scalar} принимают точку зрения \citep{rooth1985association} о семантике фокуса как о множестве контекстно заданных альтернатив.  Скалярные фокусные частицы предлагается считать пропозициональными операторами, вводящими пресуппозицию о том, что модифицируемая пропозиция \textit{прагматически сильнее} контекстных альтернатив.

\medskip

Понятие прагматической силы определяется следующим образом:

\begin{exe}
    \ex Пропозиция $ x $ \textbf{прагматически сильнее} пропозиции $ y $ (в отношении обсуждаемого вопроса $ Q $), если релевантные контекстные импликации $ x $ (в отношении $ Q $) обуславливают релевантные контекстные импликации $ y $.
\end{exe}

\textit{Контекстные импликации} --- это утверждения, выводимые одновременно из высказывания и контекста, но из высказывания или контекста по отдельности \citep{sperber2004experimental}; контекстные импликации релевантны в отношении обсуждаемого вопроса, если они дают некоторую информацию о предмете этого вопроса (либо уточняют его). 

\medskip

\citep{gast2011scalar} используют понятие контекстных импликаций как возможное решение проблем стандартного (вероятностного) анализа семантики скалярных операторов: не во всех случаях можно говорить о том, что пропозиция, модифицируемая скалярным оператором, \textit{менее вероятна} своих альтернатив (\ref{failedProbabilityAnswer}):

\begin{exe}
    \ex \begin{xlist}
    \ex \label{failedProbabilityQUD} \textit{(Обсуждаемый вопрос)} Насколько темными являются кожаные автомобильные сидения?
    \ex \label{failedProbabilityAnswer} По эстетическим соображения кожаные сидения в автомобилях чаще всего окрашиваются в темно-серый цвет, в основном даже в черный.
    \end{xlist}
\end{exe}

В примере (\ref{failedProbabilityAnswer}) невозможно говорить о том, что черная окраска сидений является менее вероятной алтернативой. 

\medskip

Ниже мы рассмотрим пример использования понятия контекстных импликаций в случае

\begin{exe}
    \ex It is more for aesthetic reasons that leather seats in automobiles are mainly coloured dark grey, indeed mostly even black.
\end{exe}

\medskip

\citep{gast2011scalar} предлагают следующую типологию аддитивных операторов:

\begin{table}[H]
\small
    \begin{exe}
    \ex \label{auweraTypology} 
        \Tree [.{\textsc{Аддитивные операторы}} [.{\textsc{Нескалярные}} ] [.{\textsc{Обобщенные}} [.{\textsc{Высокие}} ] [.{\textsc{Низкие}} [.{\textsc{Отрицательные}} ] [.{\textsc{Неотрицательные}} ] ] ] ]
    \end{exe}
\end{table}

\begin{table}[H]
    \small
    \begin{tabular}{C{2.5cm}|C{2.5cm}|C{2.5cm}|C{2.5cm}}
    \multicolumn{4}{c}{\textsc{Аддитивные операторы}}                   \\ \hline
    \textsc{Нескалярные} & \multicolumn{3}{c}{\textsc{Обобщенные}}      \\ \cline{2-4}
    ~ & \textsc{Высокие} & \multicolumn{2}{c}{\textsc{Низкие}}          \\ \cline{3-4}
    ~ & ~ & \textsc{Отрицательные} & \textsc{Неотрицательные}           \\
    \end{tabular}
\end{table}

{\footnotesize \textsc{Русский язык}}

\begin{table}[H]
    \begin{tabular}{|C{2.5cm} C{2.5cm}|C{2.5cm}|C{2.5cm}|} \hline
    \rowcolor{gray!20} \multicolumn{4}{|c|}{и}                  \\ \hline
    \rowcolor{gray!20} \multicolumn{1}{|c|}{тоже} & \multicolumn{3}{c|}{даже}        \\ \hline
    ~ & ~ & \cellcolor{gray!20} ни & \cellcolor{gray!20} хоть   \\ \hline
    \end{tabular}
\end{table}

\section{Скалярные фокусные операторы в русском языке}

\newpage
\bibliography{bibliography} 
\bibliographystyle{plainnat}

\end{document}
